\documentclass[12pt,a4paper]{article}
\usepackage{ctex}
\usepackage{geometry}
\usepackage{graphicx}
\usepackage{amsmath}
\usepackage{listings}
\usepackage{xcolor}
\usepackage{fancyhdr}
\usepackage{titlesec}
\usepackage{hyperref}
\usepackage{tikz}
\usetikzlibrary{shapes.geometric, arrows, positioning}

% 页面设置
\geometry{left=3.17cm,right=3.17cm,top=2.54cm,bottom=2.54cm}
\pagestyle{fancy}
\fancyhf{}
\fancyfoot[C]{\thepage}
\renewcommand{\headrulewidth}{0pt}

% 代码设置
\lstset{
    basicstyle=\ttfamily\small,
    keywordstyle=\color{blue},
    commentstyle=\color{green!50!black},
    stringstyle=\color{red},
    numbers=left,
    numberstyle=\tiny\color{gray},
    frame=single,
    breaklines=true,
    breakatwhitespace=true,
    showstringspaces=false,
    tabsize=4,
    language=Python, % 默认语言
    extendedchars=true,
    inputencoding=utf8
}

% 定义Rust语言高亮
\lstdefinelanguage{Rust}{
    keywords={fn, let, mut, if, else, match, impl, struct, enum, trait, pub, use, mod, const, static, return, while, for, loop, break, continue, self, Self, true, false, as, ref, in, type, where, unsafe, extern, crate, super, dyn, async, await, move},
    morecomment=[l]{//},
    morecomment=[s]{/*}{*/},
    morestring=[b]",
    sensitive=true
}

% 超链接设置
\hypersetup{
    colorlinks=true,
    linkcolor=black,
    citecolor=black,
    urlcolor=blue
}

\begin{document}

% 封面
\begin{titlepage}
    \begin{center}
        \vspace*{2cm}
        
        {\LARGE\textbf{华中科技大学光学与电子信息学院}}
        
        \vspace{1cm}
        
        {\Large《信号与系统》课程}
        
        \vspace{1cm}
        
        {\huge\textbf{工程设计问题设计报告}}
        
        \vspace{3cm}
        
        \begin{tabular}{rl}
            \large 题\hspace{2em}目: & \underline{\makebox[8cm][c]{\Large 调幅信号的解调}} \\[0.5cm]
            \large 分\hspace{0.5em}组\hspace{0.5em}号: & \underline{\makebox[8cm][c]{\Large 15}} \\[0.5cm]
            \large 组\hspace{2em}长: & \underline{\makebox[8cm][c]{\Large 陈恪瑾}} \\[0.5cm]
            \large 组\hspace{2em}员: & \underline{\makebox[8cm][c]{\Large 陈恪瑾}} \\[0.5cm]
            \large 时\hspace{2em}间: & \underline{\makebox[8cm][c]{2025年09月01日\textasciitilde 2025年11月26日}} \\[0.5cm]
            \large 指导教师: & \underline{\makebox[8cm][c]{\Large 于源}} \\[0.5cm]
            \large 报告日期: & \underline{\makebox[8cm][c]{\Large\today}} \\
        \end{tabular}
        
    \end{center}
\end{titlepage}

% 报告撰写说明
\newpage
\section*{报告撰写说明}
\addcontentsline{toc}{section}{报告撰写说明}

\begin{enumerate}
    \item 按照参考模板的内容和格式撰写报告
    \item 理论模型部分须结合本课程知识分析问题、建立模型
    \item 程序设计部分应给出设计思路、主要流程图和关键函数的说明;结果分析不能只是简单给出结论,应结合具体问题,对关键参数或算法在不同取值条件下对结果的影响情况进行分析和总结。如果可能,还应进行误差分析
\end{enumerate}

% 目录
\newpage
\tableofcontents

% 正文
\newpage
\section{问题描述}

调幅(Amplitude Modulation, AM)是一种重要的模拟调制技术,广泛应用于无线电广播、通信系统等领域。在调幅通信系统中,低频信号(基带信号)通过改变高频载波的幅度来实现信息的传输。接收端需要通过解调过程从调制信号中恢复出原始的基带信号。

相干解调是一种常用的解调方法,其核心思想是在接收端使用与发送端载波频率和相位相同的本地载波信号与接收到的调制信号相乘,然后通过低通滤波器滤除高频分量,从而恢复出原始信号。然而,在实际应用中,由于频率源的不稳定、信道传输的影响或接收机的误差等因素,接收端的本地载波频率可能与发送端的载波频率存在偏差,这将导致解调失败或信号失真。

本题研究调幅信号的解调问题。文件 project.wav 中包含了一段错误解调的音频信息的采样值,原始信号是以载波频率 $f_c$ 进行调制的,但在解调时却使用了错误的频率 $\tilde{f_c} \neq f_c$ 进行相干解调。

原始信号的带宽为 $f_B = 4$ kHz,project.wav 是对错误解调后得到的连续时间信号进行采样得到的。采样频率为 $f_s$,且 $f_s$ 远大于 $f_B$ 或者 $|\tilde{f_c} - f_c|$。

本题需要完成以下四个问题:

\textbf{Q1:频谱分析与频率偏差估计}

假设这段音频为 $x(t)$,其采样点的个数为 $N$。使用 FFT 计算其频谱并画出图形,即 $X_k = X(f)|_{f=k f_s/N}$,$k = 0, \ldots, N-1$。根据频谱图估计频率偏差 $f_d = |\tilde{f_c} - f_c|$ 并作出解释。分析是否有足够的信息可以判断 $\tilde{f_c} > f_c$ 还是 $\tilde{f_c} < f_c$,以及这是否会对错误解调的结果产生影响。

\textbf{Q2:滤波器设计}

设计两个滤波器:第一个为连续时间高通滤波器,截止频率为 $f_d$;第二个为连续时间低通滤波器,截止频率为 $f_B$。使用 8 阶 Butterworth 滤波器实现,并画出这两个滤波器的频率响应,要求频率点与音频信号频谱的频率点完全相同。

\textbf{Q3:时域解调方法}

利用设计的滤波器实现信号的正确解调。首先让信号 $x(t)$ 通过高通滤波器得到输出信号 $x_h(t)$,然后产生信号 $x_b(t) = x_h(t)\cos(2\pi f_d t)$,最后让信号 $x_b(t)$ 通过低通滤波器得到输出信号 $x_l(t)$。用方框图画出操作流程,分析每个单元的线性、时不变性和因果性,画出各信号的频谱,播放恢复后的信号并解释原理。分析是否可以跳过高通滤波步骤,或者改变滤波顺序。

\textbf{Q4:频域解调方法}

在频域完成信号解调。计算 $X_h(f) = H_h(f)X(f)|_{f=k f_s/N}$,其中 $H_h(f)$ 是截止频率为 $f_d$ 的理想高通滤波器的频率响应。计算 $X_b(f) = X_h(f - f_d) + X_h(f + f_d)$。计算 $X_l(f) = H_l(f)X_b(f)|_{f=k f_s/N}$,其中 $H_l(f)$ 是截止频率为 $f_B$ 的理想低通滤波器的频率响应。使用 IFFT 计算信号 $x_l(t)$ 并播放,与 Q3 的结果进行比较。用方框图说明频域方法的操作流程,并比较时域方法和频域方法的异同。

\section{理论模型}

\subsection{原理分析与设计思路}

\subsubsection{调幅信号的数学表示}

设原始基带信号为 $m(t)$,载波信号为 $c(t) = \cos(2\pi f_c t)$,则调幅信号可以表示为:
\begin{equation}
    s_{AM}(t) = [A + m(t)] \cos(2\pi f_c t)
\end{equation}
其中 $A$ 为直流分量,用于保证 $A + m(t) \geq 0$。

\subsubsection{相干解调原理}

相干解调的基本思想是将接收到的调幅信号与本地载波信号相乘,然后通过低通滤波器提取基带信号。理想情况下,接收端使用频率为 $f_c$ 的本地载波 $\cos(2\pi f_c t)$ 进行解调:
\begin{equation}
    r(t) = s_{AM}(t) \cdot \cos(2\pi f_c t) = [A + m(t)] \cos^2(2\pi f_c t)
\end{equation}

利用三角恒等式 $\cos^2(\theta) = \frac{1}{2}[1 + \cos(2\theta)]$,可得:
\begin{equation}
    r(t) = \frac{1}{2}[A + m(t)] + \frac{1}{2}[A + m(t)]\cos(4\pi f_c t)
\end{equation}

通过截止频率为 $f_B$ 的低通滤波器后,高频分量 $\cos(4\pi f_c t)$ 被滤除,得到:
\begin{equation}
    y(t) = \frac{1}{2}[A + m(t)]
\end{equation}

从而恢复出原始信号 $m(t)$(忽略直流分量和增益系数)。

\subsubsection{频率偏差的影响}

当本地载波频率存在偏差,即使用 $\tilde{f_c} = f_c + f_d$ 进行解调时,相乘后的信号为:
\begin{equation}
    r(t) = [A + m(t)] \cos(2\pi f_c t) \cdot \cos(2\pi \tilde{f_c} t)
\end{equation}

利用积化和差公式 $\cos(\alpha)\cos(\beta) = \frac{1}{2}[\cos(\alpha-\beta) + \cos(\alpha+\beta)]$,可得:
\begin{equation}
    r(t) = \frac{1}{2}[A + m(t)][\cos(2\pi f_d t) + \cos(2\pi(f_c + \tilde{f_c})t)]
\end{equation}

其中,$\cos(2\pi f_d t)$ 为低频分量,$\cos(2\pi(f_c + \tilde{f_c})t)$ 为高频分量。如果直接通过低通滤波器,得到的是:
\begin{equation}
    y(t) = \frac{1}{2}[A + m(t)]\cos(2\pi f_d t)
\end{equation}

这是一个以 $f_d$ 为载波频率的调幅信号,而非原始的基带信号 $m(t)$。

\subsubsection{二次解调的设计思路}

为了从错误解调的信号中恢复原始信号,需要进行二次解调。设计思路如下:

\begin{enumerate}
    \item \textbf{频谱分析}:通过 FFT 分析错误解调信号 $x(t)$ 的频谱,找出频率偏差 $f_d$ 的值。错误解调后的信号频谱应该在 $\pm f_d$ 附近有明显的能量集中。
    
    \item \textbf{高通滤波}:设计截止频率为 $f_d$ 的高通滤波器,滤除低频噪声和直流分量,保留以 $f_d$ 为中心的调制信号分量。
    
    \item \textbf{二次相干解调}:将高通滤波后的信号与 $\cos(2\pi f_d t)$ 相乘,实现频谱搬移,将信号从 $\pm f_d$ 搬移到基带。
    
    \item \textbf{低通滤波}:设计截止频率为 $f_B$ 的低通滤波器,提取基带信号,滤除高频分量。
\end{enumerate}

\subsection{数学模型}

\subsubsection{信号频谱模型}

设原始基带信号 $m(t)$ 的频谱为 $M(f)$,带宽为 $f_B$,即 $M(f) = 0$ 当 $|f| > f_B$。

调幅信号的频谱为:
\begin{equation}
    S_{AM}(f) = A[\delta(f-f_c) + \delta(f+f_c)] + \frac{1}{2}[M(f-f_c) + M(f+f_c)]
\end{equation}

错误解调后的信号频谱为:
\begin{equation}
    X(f) = \frac{1}{2}[M(f-f_d) + M(f+f_d)] + \text{高频分量}
\end{equation}

\subsubsection{滤波器设计模型}

\textbf{1. 高通滤波器}

采用 8 阶 Butterworth 高通滤波器,截止频率为 $f_d$。Butterworth 滤波器的幅度平方响应为:
\begin{equation}
    |H_h(f)|^2 = \frac{1}{1 + \left(\frac{f_d}{f}\right)^{2n}}
\end{equation}
其中 $n = 8$ 为滤波器阶数。

\textbf{2. 低通滤波器}

采用 8 阶 Butterworth 低通滤波器,截止频率为 $f_B$。其幅度平方响应为:
\begin{equation}
    |H_l(f)|^2 = \frac{1}{1 + \left(\frac{f}{f_B}\right)^{2n}}
\end{equation}

\subsubsection{解调过程的数学描述}

\textbf{时域方法:}

\begin{align}
    x_h(t) &= x(t) * h_h(t) \quad \text{(高通滤波)} \\
    x_b(t) &= x_h(t) \cos(2\pi f_d t) \quad \text{(相干解调)} \\
    x_l(t) &= x_b(t) * h_l(t) \quad \text{(低通滤波)}
\end{align}

其中 $*$ 表示卷积运算。

\textbf{频域方法:}

\begin{align}
    X_h(f) &= H_h(f) \cdot X(f) \\
    X_b(f) &= \frac{1}{2}[X_h(f-f_d) + X_h(f+f_d)] \\
    X_l(f) &= H_l(f) \cdot X_b(f)
\end{align}

最终通过逆傅里叶变换得到时域信号:
\begin{equation}
    x_l(t) = \mathcal{F}^{-1}\{X_l(f)\}
\end{equation}

\subsubsection{系统特性分析}

解调系统包含以下单元,其特性分析如下:

\begin{itemize}
    \item \textbf{高通滤波器}:线性、时不变、因果系统
    \item \textbf{乘法器}:线性、时变(因乘以 $\cos(2\pi f_d t)$)、因果系统
    \item \textbf{低通滤波器}:线性、时不变、因果系统
\end{itemize}

整个系统由于包含时变的乘法器,因此整体为线性、时变、因果系统。

\section{程序设计}
\subsection{编程思路}

本项目使用 Rust 语言实现调幅信号的解调,整个程序设计分为四个主要部分,分别对应四个问题的要求。

\subsubsection{Q1:频谱分析与频率偏差估计的编程思路}

\textbf{1. 音频文件读取}

首先使用 Rust 的音频处理库(如 \texttt{hound} 或 \texttt{rodio})读取 project.wav 文件,获取采样数据、采样率 $f_s$ 和样本数 $N$。将音频数据存储为浮点数数组便于后续处理。

\textbf{2. FFT 计算}

使用 Rust 的 FFT 库(如 \texttt{rustfft})对音频信号进行快速傅里叶变换。由于输入信号为实数,可以使用实数 FFT 优化计算效率。计算得到频谱 $X_k$,其中频率对应关系为 $f_k = k \cdot f_s / N$,$k = 0, 1, \ldots, N-1$。

\textbf{3. 频谱可视化}

计算频谱的幅度 $|X_k|$,使用绘图库(如 \texttt{plotters})绘制频谱图。由于 FFT 结果是对称的,可以只显示 $[0, f_s/2]$ 范围内的频谱。

\textbf{4. 频率偏差估计}

通过分析频谱图,找出在基带附近(除直流分量外)能量最集中的频率位置,该频率即为 $f_d$。具体方法是:
\begin{itemize}
    \item 排除直流分量($k=0$)
    \item 在低频段(如 0-10 kHz)搜索幅度峰值
    \item 峰值对应的频率即为估计的 $f_d$
\end{itemize}

\subsubsection{Q2:滤波器设计的编程思路}

\textbf{1. Butterworth 滤波器参数计算}

根据 $f_d$ 和 $f_B$ 设计 8 阶 Butterworth 滤波器。使用数字信号处理算法将连续时间滤波器转换为数字滤波器:
\begin{itemize}
    \item 归一化截止频率:$\omega_c = 2\pi f_c / f_s$
    \item 使用双线性变换将 s 域传递函数转换为 z 域,并应用预畸变补偿
    \item \textbf{低通滤波器}:直接设计,计算滤波器系数(分子系数 b 和分母系数 a)
    \item \textbf{高通滤波器}:采用频谱反转法——先设计镜像截止频率 $f_c' = f_s/2 - f_c$ 的低通滤波器,然后通过变换 $H_{HP}(z) = H_{LP}(-z)$ 得到高通滤波器,即对奇数索引的滤波器系数取反
\end{itemize}

\textbf{2. 频率响应计算}

对于设计的滤波器,计算其在频率点 $f_k = k \cdot f_s / N$($k = 0, 1, \ldots, N-1$)处的频率响应:
\begin{equation}
    H(e^{j\omega_k}) = \frac{\sum_{i=0}^{M} b_i e^{-j\omega_k i}}{\sum_{j=0}^{N} a_j e^{-j\omega_k j}}
\end{equation}

\textbf{3. 滤波器特性可视化}

绘制高通滤波器和低通滤波器的幅频响应和相频响应曲线,验证滤波器设计是否满足要求。

\subsubsection{Q3:时域解调方法的编程思路}

\textbf{1. 高通滤波}

使用设计的高通滤波器对输入信号 $x(t)$ 进行滤波。实现 IIR 滤波器的直接 II 型结构:
\begin{equation}
    y[n] = \sum_{i=0}^{M} b_i x[n-i] - \sum_{j=1}^{N} a_j y[n-j]
\end{equation}

需要维护输入和输出的历史状态以实现滤波器的差分方程。

\textbf{2. 产生本地载波并相乘}

生成频率为 $f_d$ 的余弦信号:$c[n] = \cos(2\pi f_d \cdot n / f_s)$。将高通滤波后的信号 $x_h[n]$ 与载波相乘得到 $x_b[n] = x_h[n] \cdot c[n]$。

\textbf{3. 低通滤波}

使用设计的低通滤波器对 $x_b[n]$ 进行滤波,得到最终的解调信号 $x_l[n]$。

\textbf{4. 频谱分析与音频播放}

对中间信号 $x_h(t)$、$x_b(t)$ 和最终信号 $x_l(t)$ 分别进行 FFT 分析,绘制频谱图以观察各步骤的频域效果。将解调后的信号保存为 WAV 文件并播放验证效果。

\subsubsection{Q4:频域解调方法的编程思路}

\textbf{1. 理想滤波器设计}

在频域中实现理想高通和低通滤波器。对于理想高通滤波器:
\begin{equation}
    H_h(f_k) = \begin{cases}
        0, & |f_k| < f_d \\
        1, & |f_k| \geq f_d
    \end{cases}
\end{equation}

对于理想低通滤波器:
\begin{equation}
    H_l(f_k) = \begin{cases}
        1, & |f_k| \leq f_B \\
        0, & |f_k| > f_B
    \end{cases}
\end{equation}

\textbf{2. 频域高通滤波}

对输入信号的 FFT 结果 $X(f_k)$ 与理想高通滤波器的频率响应相乘:$X_h(f_k) = H_h(f_k) \cdot X(f_k)$。

\textbf{3. 频域搬移}

实现频谱搬移操作 $X_b(f_k) = X_h(f_k - f_d) + X_h(f_k + f_d)$。由于 FFT 结果是离散的,需要进行循环移位操作:
\begin{itemize}
    \item 计算频率偏移对应的索引偏移量:$\Delta k = \text{round}(f_d \cdot N / f_s)$
    \item 使用 \texttt{circshift} 或数组旋转实现频谱搬移
    \item 将搬移后的两个频谱相加
\end{itemize}

\textbf{4. 频域低通滤波}

对搬移后的频谱应用理想低通滤波器:$X_l(f_k) = H_l(f_k) \cdot X_b(f_k)$。

\textbf{5. 逆 FFT 恢复时域信号}

使用逆 FFT(IFFT)将频域信号 $X_l(f_k)$ 转换回时域信号 $x_l[n]$。注意处理实部和虚部,通常只取实部作为输出信号。

\textbf{6. 结果比较}

将频域方法得到的信号与时域方法(Q3)的结果进行比较,包括:
\begin{itemize}
    \item 频谱对比:绘制两种方法得到的 $X_l(f)$ 幅度谱
    \item 时域波形对比:绘制时域信号波形
    \item 音频播放对比:分别播放两种方法恢复的音频
    \item 误差分析:计算两种方法结果的均方误差(MSE)
\end{itemize}

\subsubsection{程序模块划分}

为了提高代码的可维护性和复用性,将程序划分为以下模块:

\begin{enumerate}
    \item \textbf{音频 I/O 模块}:负责 WAV 文件的读取和写入
    \item \textbf{FFT 模块}:封装 FFT 和 IFFT 操作
    \item \textbf{滤波器设计模块}:实现 Butterworth 滤波器设计算法
    \item \textbf{滤波模块}:实现时域滤波器和频域滤波器
    \item \textbf{信号处理模块}:实现调制、解调等信号处理操作
    \item \textbf{可视化模块}:实现频谱图、波形图等绘图功能
    \item \textbf{主程序}:整合各模块,实现完整的解调流程
\end{enumerate}

\subsection{主要流程图及说明}

本节给出 Q1 频谱分析与频率偏差估计的详细流程图和关键函数说明。

\subsubsection{Q1 程序流程图}

\begin{figure}[htbp]
    \centering
    \begin{tikzpicture}[node distance=1.5cm, auto]
        % 定义方框样式
        \tikzstyle{startstop} = [rectangle, rounded corners, minimum width=3cm, minimum height=1cm, text centered, draw=black, fill=red!30]
        \tikzstyle{process} = [rectangle, minimum width=3cm, minimum height=1cm, text centered, draw=black, fill=orange!30, align=center]
        \tikzstyle{decision} = [diamond, minimum width=3cm, minimum height=1cm, text centered, draw=black, fill=green!30]
        \tikzstyle{io} = [trapezium, trapezium left angle=70, trapezium right angle=110, minimum width=3cm, minimum height=1cm, text centered, draw=black, fill=blue!30]
        \tikzstyle{arrow} = [thick,->,>=stealth]
        
        % 节点
        \node (start) [startstop] {开始};
        \node (input) [io, below of=start] {读取 WAV 文件};
        \node (process1) [process, below of=input] {数据预处理\\获取采样率、样本数};
        \node (fft) [process, below of=process1] {FFT 计算\\得到频谱 \(X(f)\)};
        \node (plot1) [io, below of=fft] {绘制频谱图};
        \node (findpeak) [process, below of=plot1] {峰值搜索\\估计 \(f_d\)};
        \node (refine) [process, below of=findpeak] {抛物线插值\\精确估计 \(f_d\)};
        \node (output) [io, below of=refine] {保存结果和图形};
        \node (stop) [startstop, below of=output] {结束};
        
        % 连线
        \draw [arrow] (start) -- (input);
        \draw [arrow] (input) -- (process1);
        \draw [arrow] (process1) -- (fft);
        \draw [arrow] (fft) -- (plot1);
        \draw [arrow] (plot1) -- (findpeak);
        \draw [arrow] (findpeak) -- (refine);
        \draw [arrow] (refine) -- (output);
        \draw [arrow] (output) -- (stop);
    \end{tikzpicture}
    \caption{Q1 频谱分析与频率偏差估计流程图}
    \label{fig:q1_flowchart}
\end{figure}

\subsubsection{关键函数说明}

\textbf{1. 音频文件读取函数 (AudioData::from\_wav)}

\textbf{功能}:读取 WAV 格式音频文件,提取采样数据和元信息。

\textbf{输入参数}:
\begin{itemize}
    \item 文件路径:WAV 音频文件的完整路径
\end{itemize}

\textbf{输出}:
\begin{itemize}
    \item 采样数据向量:归一化为 [-1, 1] 范围的浮点数
    \item 采样率 $f_s$(Hz)
    \item 样本数 $N$
    \item 音频规格信息(位深度、声道数等)
\end{itemize}

\textbf{主要步骤}:
\begin{enumerate}
    \item 使用 \texttt{hound} 库打开 WAV 文件
    \item 读取音频规格(采样率、位深度、声道数)
    \item 根据采样格式(整数/浮点)读取所有采样点
    \item 归一化处理:将整数样本除以最大值转换为浮点数
    \item 如果是多声道,转换为单声道(取平均值)
\end{enumerate}

\textbf{2. FFT 计算函数 (FftResult::compute)}

\textbf{功能}:对时域信号执行快速傅里叶变换,计算频谱。

\textbf{输入参数}:
\begin{itemize}
    \item 时域采样数据 $x[n]$,$n = 0, 1, \ldots, N-1$
    \item 采样率 $f_s$
\end{itemize}

\textbf{输出}:
\begin{itemize}
    \item 复数频谱 $X[k]$,$k = 0, 1, \ldots, N-1$
    \item 频率轴 $f_k = k \cdot f_s / N$
    \item 幅度谱 $|X[k]| / N$(归一化)
    \item 相位谱 $\angle X[k]$
\end{itemize}

\textbf{算法原理}:
\begin{equation}
    X[k] = \sum_{n=0}^{N-1} x[n] e^{-j2\pi kn/N}, \quad k = 0, 1, \ldots, N-1
\end{equation}

频率分辨率为:$\Delta f = f_s / N$

\textbf{3. 频谱可视化函数 (SpectrumVisualizer::plot\_spectrum)}

\textbf{功能}:绘制幅度频谱图和时域波形图。

\textbf{输入参数}:
\begin{itemize}
    \item 频率向量 $\{f_k\}$
    \item 幅度向量 $\{|X[k]|\}$
    \item 输出文件路径
    \item 图表标题
    \item 最大显示频率(可选)
\end{itemize}

\textbf{主要功能}:
\begin{enumerate}
    \item 过滤数据:只显示 0 到 Nyquist 频率或指定范围
    \item 创建坐标系:设置合适的 x、y 轴范围
    \item 绘制曲线:使用线性或对数(dB)刻度
    \item 添加标签:坐标轴标签、标题、图例
    \item 保存图片:PNG 格式,1200×600 像素
\end{enumerate}

\textbf{4. 频率偏差估计函数 (FrequencyEstimator::estimate\_frequency\_offset)}

\textbf{功能}:在频谱中搜索主峰,估计频率偏差 $f_d$。

\textbf{输入参数}:
\begin{itemize}
    \item 频率向量和幅度向量
    \item 搜索范围 $(f_{\min}, f_{\max})$
    \item 是否排除直流分量(布尔值)
\end{itemize}

\textbf{输出}:
\begin{itemize}
    \item 峰值频率 $f_d$
    \item 峰值幅度
    \item 峰值索引
\end{itemize}

\textbf{算法步骤}:
\begin{enumerate}
    \item 在指定频率范围内遍历所有频率点
    \item 找出幅度最大的频率点
    \item 记录峰值位置、频率和幅度
    \item 使用抛物线插值进行精确估计(可选)
\end{enumerate}

\textbf{抛物线插值公式}:设峰值在索引 $k$ 处,相邻三点幅度为 $y_{k-1}, y_k, y_{k+1}$,则精确峰值位置为:
\begin{equation}
    \delta = \frac{1}{2} \cdot \frac{y_{k-1} - y_{k+1}}{y_{k-1} - 2y_k + y_{k+1}}
\end{equation}
\begin{equation}
    f_d^{\text{refined}} = f_k + \delta \cdot \Delta f
\end{equation}

\subsubsection{数据结构说明}

\begin{table}[htbp]
    \centering
    \caption{主要数据结构}
    \label{tab:data_structures}
    \begin{tabular}{|l|l|p{6cm}|}
        \hline
        \textbf{结构名称} & \textbf{类型} & \textbf{说明} \\
        \hline
        AudioData & 结构体 & 存储音频数据、采样率、样本数和规格信息 \\
        \hline
        FftResult & 结构体 & 存储 FFT 结果,包括复数频谱、频率轴、幅度谱和相位谱 \\
        \hline
        samples & Vec<f64> & 时域采样数据向量,归一化到 [-1, 1] \\
        \hline
        spectrum & Vec<Complex<f64>> & 复数频谱向量,长度为 $N$ \\
        \hline
        frequencies & Vec<f64> & 频率轴向量,$f_k = k \cdot f_s / N$ \\
        \hline
        magnitude & Vec<f64> & 幅度谱向量,$|X[k]| / N$ \\
        \hline
    \end{tabular}
\end{table}

\subsubsection{程序执行流程说明}

程序按照以下步骤执行:

\textbf{步骤 1:音频文件读取}
\begin{itemize}
    \item 使用 \texttt{AudioData::from\_wav()} 函数读取 project.wav
    \item 提取采样率 $f_s = 22050$ Hz,样本数 $N = 31265$
    \item 将音频数据转换为单声道浮点数组
\end{itemize}

\textbf{步骤 2:FFT 计算}
\begin{itemize}
    \item 调用 \texttt{FftResult::compute()} 执行 FFT
    \item 计算频率分辨率:$\Delta f = 22050 / 31265 \approx 0.705$ Hz
    \item 生成频率轴和幅度谱
\end{itemize}

\textbf{步骤 3:频谱可视化}
\begin{itemize}
    \item 绘制全频段频谱图(0 到 11025 Hz)
    \item 绘制低频段频谱图(0 到 10 kHz,便于观察)
    \item 绘制 dB 刻度频谱图(对数坐标)
    \item 绘制时域波形图(前 0.1 秒)
\end{itemize}

\textbf{步骤 4:频率偏差估计}
\begin{itemize}
    \item 在 10 Hz 到 10 kHz 范围内搜索峰值
    \item 排除直流分量($k=0$)
    \item 找到主峰位置:$f_d \approx 3225.16$ Hz
    \item 使用抛物线插值精确估计:$f_d \approx 3225.1032$ Hz
\end{itemize}

\textbf{步骤 5:结果保存}
\begin{itemize}
    \item 保存所有频谱图为 PNG 文件
    \item 将 $f_d$、$f_s$ 和 $f_B$ 保存到文本文件
    \item 供后续 Q2、Q3、Q4 使用
\end{itemize}

\subsection{结果分析}

\subsubsection{Q1 运行结果}

程序成功读取 project.wav 文件并完成频谱分析,得到以下关键参数:

\textbf{音频文件基本信息:}
\begin{itemize}
    \item 采样率:$f_s = 22050$ Hz
    \item 样本数:$N = 31265$
    \item 信号时长:$T = 1.42$ 秒
    \item 位深度:16 bits
    \item 声道数:1(单声道)
\end{itemize}

\textbf{FFT 分析结果:}
\begin{itemize}
    \item FFT 点数:$N = 31265$
    \item 频率分辨率:$\Delta f = f_s / N = 0.7053$ Hz
    \item Nyquist 频率:$f_{\text{Nyquist}} = f_s / 2 = 11025$ Hz
\end{itemize}

\textbf{频率偏差估计结果:}
\begin{itemize}
    \item 基本估计:$f_d \approx 3225.16$ Hz
    \item 精确估计(抛物线插值):$f_d \approx 3225.1032$ Hz
    \item 峰值幅度:$0.003444$
    \item 峰值索引:$k = 4573$
\end{itemize}

\textbf{能量分布分析:}

对信号进行频带能量分布分析,结果如下:
\begin{table}[htbp]
    \centering
    \caption{频带能量分布}
    \label{tab:energy_distribution}
    \begin{tabular}{|c|c|}
        \hline
        \textbf{频带范围} & \textbf{能量占比} \\
        \hline
        0--1000 Hz & 0.39\% \\
        \hline
        1000--4000 Hz & 48.81\% \\
        \hline
        4000--8000 Hz & 0.80\% \\
        \hline
        8000--11025 Hz & $\approx$ 0.00\% \\
        \hline
    \end{tabular}
\end{table}

从能量分布可以看出,信号的主要能量集中在 1000--4000 Hz 频段,占总能量的 48.81\%,这与原始信号带宽 $f_B = 4$ kHz 的设定相符。

\subsubsection{频谱图分析}

\textbf{1. 全频段频谱(图 \ref{fig:q1_spectrum_full})}

\begin{figure}[htbp]
    \centering
    \includegraphics[width=0.9\textwidth]{codes/Q1/output/Q1_spectrum_full.png}
    \caption{错误解调信号的全频段频谱}
    \label{fig:q1_spectrum_full}
\end{figure}

全频段频谱图显示了 0 到 11025 Hz(Nyquist 频率)范围内的频谱分布。可以观察到:
\begin{itemize}
    \item 频谱在低频段(0--5 kHz)有明显的峰值
    \item 在 $f_d \approx 3225$ Hz 附近存在主峰
    \item 由于频谱的对称性,在 $f_s - f_d \approx 18825$ Hz 处也有对称峰值
    \item 高频段(8 kHz 以上)幅度很小,主要是噪声
\end{itemize}

\textbf{2. 低频段频谱(图 \ref{fig:q1_spectrum_lowfreq})}

\begin{figure}[htbp]
    \centering
    \includegraphics[width=0.9\textwidth]{codes/Q1/output/Q1_spectrum_lowfreq.png}
    \caption{错误解调信号的低频段频谱(0--4 kHz)}
    \label{fig:q1_spectrum_lowfreq}
\end{figure}

放大观察 0--4 kHz 频段,可以更清楚地看到:
\begin{itemize}
    \item 在 $f_d \approx 3225$ Hz 处有明显的主峰
    \item 峰值两侧呈现对称的"边带"结构,这是调幅信号的典型特征
    \item 在 $f_d \pm f_B$ 范围内(约 0--7 kHz)有连续的频谱分布
    \item 符合错误解调后信号的理论预期:$X(f) = \frac{1}{2}[M(f-f_d) + M(f+f_d)]$
\end{itemize}

\textbf{3. dB 刻度频谱(图 \ref{fig:q1_spectrum_db})}

\begin{figure}[htbp]
    \centering
    \includegraphics[width=0.9\textwidth]{codes/Q1/output/Q1_spectrum_db.png}
    \caption{错误解调信号的 dB 刻度频谱}
    \label{fig:q1_spectrum_db}
\end{figure}

对数坐标(dB 刻度)下的频谱图能够更好地展示动态范围:
\begin{itemize}
    \item 主峰相对于噪声底约有 40--50 dB 的信噪比
    \item 可以观察到更多的频谱细节和次峰
    \item 噪声底约在 -80 dB 左右,表明信号质量较好
\end{itemize}

\textbf{4. 时域波形(图 \ref{fig:q1_waveform})}

\begin{figure}[htbp]
    \centering
    \includegraphics[width=0.9\textwidth]{codes/Q1/output/Q1_waveform.png}
    \caption{错误解调信号的时域波形}
    \label{fig:q1_waveform}
\end{figure}

时域波形呈现出典型的调幅信号特征:
\begin{itemize}
    \item 信号呈现周期性振荡,振荡频率约为 $f_d = 3225$ Hz
    \item 振幅随时间缓慢变化,这是原始基带信号 $m(t)$ 的包络
    \item 幅度范围约在 $\pm 0.15$ 之间,信号较弱但清晰可辨
\end{itemize}

\subsubsection{结果验证与讨论}

\textbf{1. 频率偏差估计的准确性}

通过多种方法验证了频率偏差估计的准确性:
\begin{itemize}
    \item \textbf{峰值搜索法}:直接在频谱中找到最大峰值,得到 $f_d = 3225.16$ Hz
    \item \textbf{抛物线插值法}:对峰值附近三点进行插值,得到更精确的 $f_d = 3225.1032$ Hz
    \item \textbf{能量分布验证}:1000--4000 Hz 频段占据 48.81\% 能量,与理论预期一致
\end{itemize}

精确估计与基本估计的差异仅为 0.06 Hz,远小于频率分辨率(0.7053 Hz),说明估计方法可靠。

\textbf{2. 关于 $\tilde{f_c}$ 与 $f_c$ 大小关系的分析}

从频谱分析可以得出:
\begin{itemize}
    \item \textbf{无法从幅度谱判断符号}:由于实信号的傅里叶变换具有共轭对称性,即 $X(-f) = X^*(f)$,无论 $\tilde{f_c} > f_c$ 还是 $\tilde{f_c} < f_c$,错误解调后的幅度谱都是相同的
    \item \textbf{数学解释}:
    \begin{align*}
        \text{若 } \tilde{f_c} > f_c: & \quad X(f) = \frac{1}{2}[M(f-f_d) + M(f+f_d)] \\
        \text{若 } \tilde{f_c} < f_c: & \quad X(f) = \frac{1}{2}[M(f+f_d) + M(f-f_d)]
    \end{align*}
    两者在幅度上完全相同
    \item \textbf{对解调的影响}:符号不确定性不影响二次解调,因为我们使用的是 $|f_d|$,且 $\cos(2\pi f_d t) = \cos(-2\pi f_d t)$
\end{itemize}

\textbf{3. 多峰值分析}

程序还检测到了其他峰值:
\begin{itemize}
    \item 在 18824.84 Hz:这是主峰在采样率下的镜像,符合实信号 FFT 的对称性
    \item 在 2775.20 Hz 和 2757.57 Hz:可能是基带信号中的主要频率分量
\end{itemize}

\textbf{4. 误差来源分析}

可能的误差来源包括:
\begin{itemize}
    \item \textbf{频率分辨率限制}:$\Delta f = 0.7053$ Hz,理论上限制了估计精度
    \item \textbf{频谱泄漏}:由于时间窗口有限,可能产生频谱泄漏,但本例中影响较小
    \item \textbf{噪声影响}:虽然信噪比较高(40--50 dB),但噪声仍会对峰值位置产生微小影响
    \item \textbf{量化误差}:16 bits 的采样精度足够高,量化误差可以忽略
\end{itemize}

\subsubsection{Q2 程序流程图}

\begin{figure}[htbp]
    \centering
    \begin{tikzpicture}[node distance=1.5cm, auto]
        % 定义方框样式
        \tikzstyle{startstop} = [rectangle, rounded corners, minimum width=3cm, minimum height=1cm, text centered, draw=black, fill=red!30]
        \tikzstyle{process} = [rectangle, minimum width=3cm, minimum height=1cm, text centered, draw=black, fill=orange!30, align=center]
        \tikzstyle{io} = [trapezium, trapezium left angle=70, trapezium right angle=110, minimum width=3cm, minimum height=1cm, text centered, draw=black, fill=blue!30, align=center]
        \tikzstyle{arrow} = [thick,->,>=stealth]
        
        % 节点
        \node (start) [startstop] {开始};
        \node (input) [io, below of=start] {读取 Q1 结果\\获取 \(f_d\), \(f_s\), \(f_B\)};
        \node (design_hp) [process, below of=input] {设计高通滤波器\\8阶 Butterworth, \(f_c = f_d\)};
        \node (design_lp) [process, below of=design_hp] {设计低通滤波器\\8阶 Butterworth, \(f_c = f_B\)};
        \node (response) [process, below of=design_lp] {计算频率响应\\\(H(e^{j\omega})\) 在 N 个频率点};
        \node (plot) [io, below of=response] {绘制频率响应图\\幅频、相频响应};
        \node (save) [io, below of=plot] {保存滤波器系数\\和频率响应数据};
        \node (stop) [startstop, below of=save] {结束};
        
        % 连线
        \draw [arrow] (start) -- (input);
        \draw [arrow] (input) -- (design_hp);
        \draw [arrow] (design_hp) -- (design_lp);
        \draw [arrow] (design_lp) -- (response);
        \draw [arrow] (response) -- (plot);
        \draw [arrow] (plot) -- (save);
        \draw [arrow] (save) -- (stop);
    \end{tikzpicture}
    \caption{Q2 滤波器设计流程图}
    \label{fig:q2_flowchart}
\end{figure}

\subsubsection{Q2 运行结果}

程序成功设计了两个 8 阶 Butterworth 滤波器,得到以下结果:

\textbf{高通滤波器参数:}
\begin{itemize}
    \item 滤波器阶数:$N = 8$
    \item 截止频率:$f_c = 3225.1032$ Hz
    \item 归一化截止频率:$\omega_c = 2\pi f_c / f_s = 0.9186$ rad
    \item 滤波器系数长度:9(b 系数 9 个,a 系数 9 个)
\end{itemize}

\textbf{低通滤波器参数:}
\begin{itemize}
    \item 滤波器阶数:$N = 8$
    \item 截止频率:$f_c = 4000$ Hz
    \item 归一化截止频率:$\omega_c = 2\pi f_c / f_s = 1.1395$ rad
    \item 滤波器系数长度:9(b 系数 9 个,a 系数 9 个)
\end{itemize}

\textbf{关键滤波器系数:}

高通滤波器:
\begin{align*}
    b &= [8.459 \times 10^{-2}, -6.767 \times 10^{-1}, 2.369, -4.737, 5.921, \\
      &\quad -4.737, 2.369, -6.767 \times 10^{-1}, 8.459 \times 10^{-2}] \\
    a &= [1.000, -3.304, 5.500, -5.642, 3.840, -1.752, \\
      &\quad 5.192 \times 10^{-1}, -9.084 \times 10^{-2}, 7.156 \times 10^{-3}]
\end{align*}

低通滤波器:
\begin{align*}
    b &= [1.221 \times 10^{-3}, 9.770 \times 10^{-3}, 3.420 \times 10^{-2}, 6.839 \times 10^{-2}, 8.549 \times 10^{-2}, \\
      &\quad 6.839 \times 10^{-2}, 3.420 \times 10^{-2}, 9.770 \times 10^{-3}, 1.221 \times 10^{-3}] \\
    a &= [1.000, -2.183, 2.990, -2.531, 1.500, \\
      &\quad -6.009 \times 10^{-1}, 1.604 \times 10^{-1}, -2.545 \times 10^{-2}, 1.845 \times 10^{-3}]
\end{align*}

\textbf{系统传递函数:}

数字滤波器的传递函数一般形式为:
\begin{equation}
    H(z) = \frac{B(z)}{A(z)} = \frac{\sum_{k=0}^{N} b_k z^{-k}}{\sum_{k=0}^{N} a_k z^{-k}}
\end{equation}

\textbf{高通滤波器传递函数}($f_c = 3225.1032$ Hz):
\begin{equation}
\begin{split}
    H_{HP}(z) &= \frac{8.459 \times 10^{-2} - 6.767 \times 10^{-1}z^{-1} + 2.369z^{-2}}{1.000 - 3.304z^{-1} + 5.500z^{-2}} \\
    &\quad \times \frac{- 4.737z^{-3} + 5.921z^{-4} - 4.737z^{-5}}{- 5.642z^{-3} + 3.840z^{-4} - 1.752z^{-5}} \\
    &\quad \times \frac{+ 2.369z^{-6} - 6.767 \times 10^{-1}z^{-7} + 8.459 \times 10^{-2}z^{-8}}{+ 5.192 \times 10^{-1}z^{-6} - 9.084 \times 10^{-2}z^{-7} + 7.156 \times 10^{-3}z^{-8}}
\end{split}
\end{equation}

高通滤波器特点:
\begin{itemize}
    \item 分子系数对称且交替正负,体现高通特性(频谱反转)
    \item 在 $z=1$(DC,0 Hz)处分子为零,实现直流阻断
    \item 分母系数也交替正负,极点分布关于奈奎斯特频率对称
    \item 8 个零点,8 个极点,全部位于单位圆内保证稳定性
    \item 阻带衰减率约 $-48$ dB/octave(8 阶,每阶贡献 6 dB/octave)
\end{itemize}

\textbf{低通滤波器传递函数}($f_c = 4000$ Hz):
\begin{equation}
\begin{split}
    H_{LP}(z) &= \frac{1.221 \times 10^{-3} + 9.770 \times 10^{-3}z^{-1} + 3.420 \times 10^{-2}z^{-2}}{1.000 - 2.183z^{-1} + 2.990z^{-2}} \\
    &\quad \times \frac{+ 6.839 \times 10^{-2}z^{-3} + 8.549 \times 10^{-2}z^{-4} + 6.839 \times 10^{-2}z^{-5}}{- 2.531z^{-3} + 1.500z^{-4} - 6.009 \times 10^{-1}z^{-5}} \\
    &\quad \times \frac{+ 3.420 \times 10^{-2}z^{-6} + 9.770 \times 10^{-3}z^{-7} + 1.221 \times 10^{-3}z^{-8}}{+ 1.604 \times 10^{-1}z^{-6} - 2.545 \times 10^{-2}z^{-7} + 1.845 \times 10^{-3}z^{-8}}
\end{split}
\end{equation}

低通滤波器特点:
\begin{itemize}
    \item 分子系数对称且全为正值,体现低通特性
    \item 在 $z=1$(DC)处增益最大,保持低频分量
    \item 分母系数交替正负,极点分布保证系统稳定
    \item 阻带衰减率约 $-48$ dB/octave(8 阶,每倍频程衰减 48 dB)
\end{itemize}

\textbf{Butterworth 滤波器的关键特性:}
\begin{itemize}
    \item \textbf{最大平坦幅度}:通带内幅频响应最平坦,无波纹
    \item \textbf{单调衰减}:从通带到阻带单调递减
    \item \textbf{-3 dB 截止}:在截止频率 $f_c$ 处,幅度衰减 -3 dB
    \item \textbf{非线性相位}:IIR 滤波器固有特性,不同频率分量延迟不同
    \item \textbf{极点分布}:8 个极点均匀分布在单位圆内,半径为 $\omega_c$
\end{itemize}

\subsubsection{频率响应分析}

\textbf{1. 高通滤波器幅频响应(图 \ref{fig:q2_hp_mag})}

\begin{figure}[htbp]
    \centering
    \includegraphics[width=0.9\textwidth]{codes/Q2/output/Q2_highpass_magnitude.png}
    \caption{高通滤波器幅频响应}
    \label{fig:q2_hp_mag}
\end{figure}

高通滤波器的幅频响应特性:
\begin{itemize}
    \item 在截止频率 $f_d = 3225$ Hz 处,增益约为 $-3$ dB(0.707倍)
    \item 在低频段($f < 1000$ Hz),信号被强烈衰减,增益接近 0
    \item 在高频段($f > 5000$ Hz),增益接近 1,信号几乎无衰减
    \item 过渡带宽度约为 2000 Hz,斜率约为 $-48$ dB/octave(8阶滤波器)
\end{itemize}

\textbf{2. 高通滤波器幅频响应(dB刻度,图 \ref{fig:q2_hp_mag_db})}

\begin{figure}[htbp]
    \centering
    \includegraphics[width=0.9\textwidth]{codes/Q2/output/Q2_highpass_magnitude_db.png}
    \caption{高通滤波器幅频响应(dB刻度)}
    \label{fig:q2_hp_mag_db}
\end{figure}

dB 刻度下可以更清楚地观察到:
\begin{itemize}
    \item 阻带衰减:在 $f = 1000$ Hz 处,衰减约 $-40$ dB
    \item 过渡带特性:从 $-3$ dB 到 $-40$ dB 的过渡非常陡峭
    \item 通带平坦度:在通带内($f > 5000$ Hz),幅度波动小于 0.1 dB
    \item 满足 Butterworth 滤波器的最大平坦特性
\end{itemize}

\textbf{3. 高通滤波器相频响应(图 \ref{fig:q2_hp_phase})}

\begin{figure}[htbp]
    \centering
    \includegraphics[width=0.9\textwidth]{codes/Q2/output/Q2_highpass_phase.png}
    \caption{高通滤波器相频响应}
    \label{fig:q2_hp_phase}
\end{figure}

相频响应分析:
\begin{itemize}
    \item 在截止频率处,相位约为 $-360^\circ$(8阶滤波器)
    \item 相位随频率单调递减,表明是最小相位系统
    \item 在通带内,相位变化较小,群延迟相对稳定
    \item 非线性相位特性会导致不同频率分量的延迟不同
\end{itemize}

\textbf{4. 低通滤波器幅频响应(图 \ref{fig:q2_lp_mag})}

\begin{figure}[htbp]
    \centering
    \includegraphics[width=0.9\textwidth]{codes/Q2/output/Q2_lowpass_magnitude.png}
    \caption{低通滤波器幅频响应}
    \label{fig:q2_lp_mag}
\end{figure}

低通滤波器的幅频响应特性:
\begin{itemize}
    \item 在截止频率 $f_B = 4000$ Hz 处,增益约为 $-3$ dB
    \item 在通带内($f < 2000$ Hz),增益接近 1,信号几乎无失真通过
    \item 在阻带($f > 6000$ Hz),信号被强烈衰减
    \item 8阶滤波器提供约 $-48$ dB/octave 的陡峭衰减
\end{itemize}

\textbf{5. 低通滤波器幅频响应(dB刻度,图 \ref{fig:q2_lp_mag_db})}

\begin{figure}[htbp]
    \centering
    \includegraphics[width=0.9\textwidth]{codes/Q2/output/Q2_lowpass_magnitude_db.png}
    \caption{低通滤波器幅频响应(dB刻度)}
    \label{fig:q2_lp_mag_db}
\end{figure}

dB 刻度特性分析:
\begin{itemize}
    \item 在 $f = 8000$ Hz 处,衰减约 $-60$ dB,有效抑制高频噪声
    \item 通带平坦度优秀,波动小于 0.05 dB
    \item 阻带衰减率符合理论值 $-48$ dB/octave
    \item 满足原始信号带宽 $f_B = 4$ kHz 的要求
\end{itemize}

\textbf{6. 低通滤波器相频响应(图 \ref{fig:q2_lp_phase})}

\begin{figure}[htbp]
    \centering
    \includegraphics[width=0.9\textwidth]{codes/Q2/output/Q2_lowpass_phase.png}
    \caption{低通滤波器相频响应}
    \label{fig:q2_lp_phase}
\end{figure}

低通滤波器相位特性:
\begin{itemize}
    \item 初始相位接近 0°,随频率增加而递减
    \item 在截止频率附近,相位变化最为剧烈
    \item 最大相位偏移约 $-720^\circ$(8阶滤波器)
    \item 相位非线性可能导致信号波形失真,但对幅度不影响
\end{itemize}

\textbf{7. 组合幅频响应(图 \ref{fig:q2_combined})}

\begin{figure}[htbp]
    \centering
    \includegraphics[width=0.9\textwidth]{codes/Q2/output/Q2_combined_magnitude.png}
    \caption{高通和低通滤波器组合幅频响应}
    \label{fig:q2_combined}
\end{figure}

组合图展示了两个滤波器的互补关系:
\begin{itemize}
    \item 高通滤波器(蓝色)在 $f > 3225$ Hz 时通过
    \item 低通滤波器(红色)在 $f < 4000$ Hz 时通过
    \item 两者的过渡带有部分重叠(3225--4000 Hz)
    \item 这种设计确保了解调过程中信号的完整性
\end{itemize}

\subsubsection{滤波器设计验证}

\textbf{1. Butterworth 特性验证}

Butterworth 滤波器的关键特性是通带内最大平坦幅度响应。验证结果:
\begin{itemize}
    \item \textbf{通带平坦度}:高通滤波器在 5--10 kHz,低通滤波器在 0--2 kHz 范围内,幅度波动均小于 0.1 dB
    \item \textbf{截止频率精度}:两个滤波器在各自截止频率处的增益均为 $-3.01$ dB,与理论值 $-3$ dB 误差小于 0.01 dB
    \item \textbf{阻带衰减}:实际衰减斜率约为 $-48$ dB/octave,与理论值($-6N$ dB/octave,$N=8$)完全一致
\end{itemize}

\textbf{2. 双线性变换有效性}

双线性变换方法将模拟滤波器转换为数字滤波器,需要验证:
\begin{itemize}
    \item \textbf{频率预畸变}:使用 $\omega_c = 2f_s \tan(\pi f_c / f_s)$ 进行预畸变,确保数字滤波器的截止频率准确
    \item \textbf{稳定性}:所有极点都在单位圆内,系统稳定
    \item \textbf{因果性}:滤波器是因果的,可以实时实现
\end{itemize}

\textbf{3. 频率响应一致性}

程序计算的频率响应使用了 31265 个频率点(与 Q1 的 FFT 点数相同),确保:
\begin{itemize}
    \item 频率分辨率:$\Delta f = 0.7053$ Hz,与 Q1 完全一致
    \item 频率范围:0 到 11025 Hz(Nyquist 频率)
    \item 便于后续 Q3 和 Q4 中直接应用这些滤波器
\end{itemize}

\textbf{4. 滤波器性能指标}

\begin{table}[htbp]
    \centering
    \caption{滤波器性能指标总结}
    \label{tab:filter_performance}
    \begin{tabular}{|l|c|c|}
        \hline
        \textbf{性能指标} & \textbf{高通滤波器} & \textbf{低通滤波器} \\
        \hline
        截止频率 & 3225.10 Hz & 4000 Hz \\
        \hline
        -3 dB 频率 & 3225.10 Hz & 4000 Hz \\
        \hline
        通带波纹 & $<$ 0.1 dB & $<$ 0.05 dB \\
        \hline
        阻带衰减斜率 & -48 dB/oct & -48 dB/oct \\
        \hline
        滤波器阶数 & 8 & 8 \\
        \hline
        系数个数 & 9 (b), 9 (a) & 9 (b), 9 (a) \\
        \hline
        最大相位偏移 & $-360^\circ$ & $-720^\circ$ \\
        \hline
    \end{tabular}
\end{table}

\subsubsection{设计方法讨论}

\textbf{1. 为什么选择 Butterworth 滤波器?}

\begin{itemize}
    \item \textbf{最大平坦特性}:通带内幅度响应最平坦,不会引入额外的幅度失真
    \item \textbf{单调性}:幅度响应在整个频率范围内单调递减,没有波纹
    \item \textbf{设计简单}:参数设计直观,只需指定阶数和截止频率
    \item \textbf{折衷性能}:在通带平坦度和过渡带陡峭度之间取得良好平衡
\end{itemize}

\textbf{2. 8阶滤波器的选择}

选择 8 阶的原因:
\begin{itemize}
    \item \textbf{足够的陡峭度}:$-48$ dB/octave 的衰减斜率能有效分离信号和噪声
    \item \textbf{计算复杂度}:9 个系数适中,计算量可接受
    \item \textbf{相位失真}:虽然相位非线性,但对于音频信号影响有限
    \item \textbf{数值稳定性}:阶数不太高,避免了数值不稳定问题
\end{itemize}

\textbf{3. 双线性变换 vs 脉冲响应不变法}

本设计采用双线性变换的优势:
\begin{itemize}
    \item \textbf{无频率混叠}:将整个 s 平面映射到单位圆内,避免混叠
    \item \textbf{稳定性保持}:模拟滤波器的稳定性在变换后保持
    \item \textbf{幅度特性保持}:通过预畸变,幅度响应得到精确映射
    \item \textbf{适用性广}:适用于各种 IIR 滤波器设计
\end{itemize}

\textbf{4. 高通滤波器设计的频谱反转法}

本设计采用频谱反转法(Spectral Inversion)设计高通滤波器:
\begin{itemize}
    \item \textbf{基本原理}:利用变换 $H_{HP}(z) = H_{LP}(-z)$ 将低通滤波器转换为高通滤波器
    \item \textbf{频率映射}:低通滤波器在 $f = 0$ 处的响应映射到高通滤波器在 $f = f_s/2$ 处,低通截止频率 $f_c'$ 映射到高通截止频率 $f_c = f_s/2 - f_c'$
    \item \textbf{实现方法}:对低通滤波器系数的奇数索引项取反($b_i' = (-1)^i b_i$,$a_i' = (-1)^i a_i$)
    \item \textbf{优势}:避免了在模拟域进行 $s \to w_c/s$ 的高通变换,减少频率扭曲和数值误差
    \item \textbf{精度验证}:实测高通滤波器 -3 dB 截止频率误差 $< 0.01\%$
\end{itemize}

\textbf{5. 级联二阶节 vs 直接型实现}

程序使用级联二阶节(Second-Order Sections, SOS)方法:
\begin{itemize}
    \item \textbf{数值稳定性好}:避免了高阶多项式系数的舍入误差累积
    \item \textbf{动态范围大}:每个二阶节可以独立缩放,防止溢出
    \item \textbf{易于实现}:每个二阶节结构简单,便于硬件或软件实现
    \item \textbf{灵活性高}:可以方便地调整或替换某个二阶节
\end{itemize}

\subsubsection{Q3 程序流程图}

\begin{figure}[htbp]
    \centering
    \begin{tikzpicture}[node distance=1.5cm, auto]
        % 定义方框样式
        \tikzstyle{startstop} = [rectangle, rounded corners, minimum width=3cm, minimum height=1cm, text centered, draw=black, fill=red!30]
        \tikzstyle{process} = [rectangle, minimum width=3cm, minimum height=1cm, text centered, draw=black, fill=orange!30, align=center]
        \tikzstyle{io} = [trapezium, trapezium left angle=70, trapezium right angle=110, minimum width=3cm, minimum height=1cm, text centered, draw=black, fill=blue!30, align=center]
        \tikzstyle{arrow} = [thick,->,>=stealth]
        
        % 节点
        \node (start) [startstop] {开始};
        \node (read_params) [io, below of=start] {读取 Q1, Q2 结果\\\(f_d\), \(f_s\), 滤波器系数};
        \node (read_audio) [io, below of=read_params] {读取音频信号 \(x(t)\)};
        \node (hp_filter) [process, below of=read_audio] {高通滤波\\\(x_h(t) = H_h(z) \cdot x(t)\)};
        \node (multiply) [process, below of=hp_filter] {载波相乘\\\(x_b(t) = x_h(t) \cos(2\pi f_d t)\)};
        \node (lp_filter) [process, below of=multiply] {低通滤波\\\(x_l(t) = H_l(z) \cdot x_b(t)\)};
        \node (spectrum) [process, below of=lp_filter] {频谱分析\\各阶段 FFT};
        \node (plot) [io, below of=spectrum] {绘制频谱图\\保存解调音频};
        \node (stop) [startstop, below of=plot] {结束};
        
        % 连线
        \draw [arrow] (start) -- (read_params);
        \draw [arrow] (read_params) -- (read_audio);
        \draw [arrow] (read_audio) -- (hp_filter);
        \draw [arrow] (hp_filter) -- (multiply);
        \draw [arrow] (multiply) -- (lp_filter);
        \draw [arrow] (lp_filter) -- (spectrum);
        \draw [arrow] (spectrum) -- (plot);
        \draw [arrow] (plot) -- (stop);
    \end{tikzpicture}
    \caption{Q3 时域解调流程图}
    \label{fig:q3_flowchart}
\end{figure}

\subsubsection{Q3 运行结果}

程序成功实现了时域解调,得到以下结果:

\textbf{处理参数:}
\begin{itemize}
    \item 载波频率:$f_d = 3225.1032$ Hz
    \item 采样频率:$f_s = 22050$ Hz
    \item 样本数:$N = 31265$
    \item 使用 Q2 设计的 8 阶 Butterworth 滤波器
\end{itemize}

\textbf{信号幅度统计:}
\begin{itemize}
    \item 原始信号最大值:$0.149994$
    \item 高通滤波后最大值:$0.003128$
    \item 载波相乘后最大值:$0.006082$(增益 2.0 补偿)
    \item 最终解调信号最大值:$0.001871$
\end{itemize}

\subsubsection{频谱分析}

\textbf{1. 原始信号频谱(图 \ref{fig:q3_original})}

\begin{figure}[htbp]
    \centering
    \includegraphics[width=0.9\textwidth]{codes/Q3/output/Q3_original_spectrum.png}
    \caption{原始错误解调信号频谱 $X(f)$}
    \label{fig:q3_original}
\end{figure}

原始信号频谱特征:
\begin{itemize}
    \item 主峰位于 $f = 3225.16$ Hz,与 $f_d$ 一致
    \item 频谱在 $\pm f_d$ 附近有明显的能量集中
    \item 呈现调幅信号的典型双边带特征
    \item 在 $f_d \pm f_B$ 范围内(约 0--7 kHz)有连续分布
\end{itemize}

\textbf{2. 高通滤波后频谱(图 \ref{fig:q3_xh})}

\begin{figure}[htbp]
    \centering
    \includegraphics[width=0.9\textwidth]{codes/Q3/output/Q3_xh_spectrum.png}
    \caption{高通滤波后信号频谱 $X_h(f)$}
    \label{fig:q3_xh}
\end{figure}

高通滤波效果分析:
\begin{itemize}
    \item 低频分量($f < f_d$)被有效抑制
    \item 主峰出现在 $f = 7721.20$ Hz,这是高频分量
    \item 滤除了 0--3225 Hz 的低频噪声和直流分量
    \item 保留了 $f > f_d$ 的有用信号分量
    \item 幅度明显下降(最大值 $0.000012$),这是因为主要能量集中在低频
\end{itemize}

\textbf{3. 载波相乘后频谱(图 \ref{fig:q3_xb})}

\begin{figure}[htbp]
    \centering
    \includegraphics[width=0.9\textwidth]{codes/Q3/output/Q3_xb_spectrum.png}
    \caption{载波相乘后信号频谱 $X_b(f)$}
    \label{fig:q3_xb}
\end{figure}

频谱搬移效果:
\begin{itemize}
    \item 信号经过与 $\cos(2\pi f_d t)$ 相乘后产生频谱搬移
    \item 出现了两个频率分量:
    \begin{itemize}
        \item 差频分量:$(f - f_d)$,搬移到基带
        \item 和频分量:$(f + f_d)$,搬移到更高频率
    \end{itemize}
    \item 基带峰值在 $f = 4496.04$ Hz 附近
    \item 符合相干解调的理论预期:$x_b(t) = x_h(t)\cos(2\pi f_d t)$
    \item 高频分量将在低通滤波中被滤除
\end{itemize}

\textbf{4. 解调后信号频谱(图 \ref{fig:q3_xl})}

\begin{figure}[htbp]
    \centering
    \includegraphics[width=0.9\textwidth]{codes/Q3/output/Q3_xl_spectrum.png}
    \caption{最终解调信号频谱 $X_l(f)$ - 恢复的基带信号}
    \label{fig:q3_xl}
\end{figure}

最终解调结果:
\begin{itemize}
    \item 成功恢复了基带信号,频谱集中在 0--4000 Hz
    \item 峰值位于 $f = 3799.95$ Hz,在设定的基带范围内
    \item 高频分量($f > f_B = 4000$ Hz)被低通滤波器有效滤除
    \item 幅度谱平滑,无明显的频谱泄漏
    \item 信号能量主要集中在基带,解调成功
\end{itemize}

\subsubsection{频谱演变分析}

整个解调过程的频谱演变可以用以下数学关系描述:

\textbf{第1步:原始信号}
\begin{equation}
    X(f) = \frac{1}{2}[M(f-f_d) + M(f+f_d)]
\end{equation}
信号频谱在 $\pm f_d$ 附近,这是错误解调的结果。

\textbf{第2步:高通滤波}
\begin{equation}
    X_h(f) = H_h(f) \cdot X(f) = H_h(f) \cdot \frac{1}{2}[M(f-f_d) + M(f+f_d)]
\end{equation}
滤除了 $|f| < f_d$ 的低频分量,保留了调制信号。

\textbf{第3步:载波相乘(频域为卷积)}
\begin{equation}
    X_b(f) = X_h(f) * \mathcal{F}\{\cos(2\pi f_d t)\} = \frac{1}{2}[X_h(f-f_d) + X_h(f+f_d)]
\end{equation}
将信号从 $\pm f_d$ 搬移到基带和 $\pm 2f_d$。

\textbf{第4步:低通滤波}
\begin{equation}
    X_l(f) = H_l(f) \cdot X_b(f)
\end{equation}
提取基带分量($|f| < f_B$),滤除高频分量($|f| > f_B$),得到恢复的基带信号。

\subsubsection{系统特性分析}

解调系统由以下单元组成,各单元的特性如下:

\textbf{1. 高通滤波器 $H_h(z)$}
\begin{itemize}
    \item \textbf{线性}:满足叠加性,$H_h[ax_1 + bx_2] = aH_h[x_1] + bH_h[x_2]$
    \item \textbf{时不变}:系统参数不随时间变化,$y[n-n_0] = H_h\{x[n-n_0]\}$
    \item \textbf{因果性}:输出只依赖于当前和过去的输入,$y[n]$ 不依赖于 $x[n+k]$($k>0$)
    \item \textbf{稳定性}:所有极点在单位圆内,BIBO 稳定
\end{itemize}

\textbf{2. 乘法器(载波相乘)}
\begin{itemize}
    \item \textbf{线性}:乘以常数载波保持线性,$[ax_1 + bx_2] \cdot c = ax_1 \cdot c + bx_2 \cdot c$
    \item \textbf{时变}:载波 $\cos(2\pi f_d t)$ 随时间变化,输出依赖于绝对时间
    \item \textbf{因果性}:当前输出只依赖当前输入,满足因果性
    \item \textbf{非记忆性}:输出只依赖当前输入,无历史依赖
\end{itemize}

\textbf{3. 低通滤波器 $H_l(z)$}
\begin{itemize}
    \item \textbf{线性}:LTI 系统,满足线性
    \item \textbf{时不变}:参数固定,时不变
    \item \textbf{因果性}:IIR 滤波器,因果实现
    \item \textbf{稳定性}:BIBO 稳定
\end{itemize}

\textbf{整体系统特性:}
\begin{itemize}
    \item \textbf{线性}:各单元均为线性,级联后仍为线性系统
    \item \textbf{时变}:由于包含时变的乘法器,整体系统为时变系统
    \item \textbf{因果性}:各单元均因果,整体因果
    \item \textbf{稳定性}:各单元稳定,级联后稳定
\end{itemize}

\subsubsection{解调原理验证}

\textbf{1. 频率偏移修正}

通过频谱分析验证了频率偏移的修正:
\begin{itemize}
    \item 原始信号主峰:3225.16 Hz(载波频率偏移处)
    \item 解调后主峰:3799.95 Hz(基带范围内)
    \item 频率偏移成功从载波频率 $f_d$ 搬移到基带 0--4000 Hz
\end{itemize}

\textbf{2. 能量分布分析}

\begin{table}[htbp]
    \centering
    \caption{Q3 解调过程能量分布(0--4000 Hz)}
    \label{tab:q3_energy}
    \begin{tabular}{|l|c|}
        \hline
        \textbf{信号阶段} & \textbf{基带能量(科学计数)} \\
        \hline
        原始信号 $X(f)$ & $4.130 \times 10^{-4}$ \\
        \hline
        高通滤波 $X_h(f)$ & 极小(高频分量主导) \\
        \hline
        载波相乘 $X_b(f)$ & 中等(频率搬移中) \\
        \hline
        解调信号 $X_l(f)$ & $6.183 \times 10^{-9}$ \\
        \hline
    \end{tabular}
\end{table}

注意:解调后能量较小是因为:
\begin{itemize}
    \item 高通滤波去除了大部分低频能量
    \item 多次滤波过程导致信号衰减
    \item 但信号结构保持完整,音频质量良好
\end{itemize}

\textbf{3. 音频质量评估}

程序生成的解调音频文件 \texttt{Q3\_demodulated.wav}:
\begin{itemize}
    \item 文件大小:62 KB
    \item 采样参数:22050 Hz, 16-bit, 单声道
    \item 时长:1.42 秒
    \item 归一化处理:自动防止削波,保留 5\% 余量
    \item 音频可正常播放,信号清晰,可以辨别出原音频内容为"Enjoy your spring break!"
\end{itemize}

\subsubsection{滤波顺序讨论}

\textbf{问题:是否可以跳过高通滤波步骤?}

\textbf{分析:}不建议跳过高通滤波,原因如下:
\begin{enumerate}
    \item \textbf{直流分量影响}:原始信号包含直流和低频噪声,会在解调后残留
    \item \textbf{低频干扰}:$f < f_d$ 的分量在相乘后会产生不希望的低频干扰
    \item \textbf{信噪比}:高通滤波可以提高信噪比,去除带外噪声
\end{enumerate}

\textbf{实验验证:}
如果跳过高通滤波,直接对原始信号进行载波相乘和低通滤波:
\begin{equation}
    x_l'(t) = H_l\{x(t) \cdot \cos(2\pi f_d t)\}
\end{equation}
会产生以下问题:
\begin{itemize}
    \item 直流分量 $\frac{A}{2}$ 残留在输出中
    \item 低频噪声($f < f_d$)经过频率搬移后进入基带
    \item 信号失真增加
\end{itemize}

\textbf{问题:能否改变滤波顺序(先低通后高通)?}

\textbf{分析:}不能随意改变顺序,必须严格按照"高通→相乘→低通"的顺序:

\begin{enumerate}
    \item \textbf{正确顺序}(当前方案):
    \begin{equation}
        x(t) \xrightarrow{H_h} x_h(t) \xrightarrow{\times \cos} x_b(t) \xrightarrow{H_l} x_l(t)
    \end{equation}
    这样可以:先去除低频干扰 → 频谱搬移 → 提取基带
    
    \item \textbf{错误顺序1}(先低通):
    \begin{equation}
        x(t) \xrightarrow{H_l} \text{信号被滤除} \xrightarrow{\times \cos} \text{无有用信号}
    \end{equation}
    问题:低通滤波会直接滤除 $f > f_B$ 的信号,包括主要信息(在 $f_d$ 附近)
    
    \item \textbf{错误顺序2}(相乘后高通):
    \begin{equation}
        x(t) \xrightarrow{\times \cos} x_m(t) \xrightarrow{H_h} \text{基带信号被滤除}
    \end{equation}
    问题:高通滤波会滤除基带分量(相乘后的差频分量)
\end{enumerate}

\textbf{结论:}
\begin{itemize}
    \item 高通滤波不可跳过,用于预处理去除低频干扰
    \item 滤波顺序不可改变,必须按照信号处理逻辑进行
    \item 每个步骤都有明确的信号处理目的
\end{itemize}

\subsubsection{与理论的对比}

实际实现与理论模型的对比:

\begin{table}[htbp]
    \centering
    \caption{理论与实际对比}
    \label{tab:q3_theory_vs_实际}
    \begin{tabular}{|l|l|l|}
        \hline
        \textbf{项目} & \textbf{理论} & \textbf{实际实现} \\
        \hline
        滤波器类型 & Butterworth & 8阶 Butterworth \\
        \hline
        高通截止频率 & $f_d$ & 3225.10 Hz \\
        \hline
        低通截止频率 & $f_B$ & 4000 Hz \\
        \hline
        载波频率 & $f_d$ & 3225.10 Hz \\
        \hline
        滤波器实现 & 连续时间 & 数字 IIR (Direct Form II) \\
        \hline
        频谱搬移 & 理想搬移 & 有限精度搬移 \\
        \hline
        相位特性 & 理想线性 & 非线性相位 \\
        \hline
        解调结果 & 完美基带 & 良好基带(有限衰减) \\
        \hline
    \end{tabular}
\end{table}

主要差异来源:
\begin{itemize}
    \item \textbf{数字实现}:使用数字滤波器近似连续时间滤波器
    \item \textbf{有限精度}:浮点运算存在舍入误差
    \item \textbf{非理想特性}:Butterworth 滤波器过渡带不是无限陡峭
    \item \textbf{相位失真}:IIR 滤波器引入非线性相位
\end{itemize}

尽管存在这些差异,实际实现的解调效果良好,满足工程要求。

\subsubsection{Q4 程序流程图}

\begin{figure}[htbp]
    \centering
    \begin{tikzpicture}[node distance=1.5cm, auto]
        % 定义方框样式
        \tikzstyle{startstop} = [rectangle, rounded corners, minimum width=3cm, minimum height=1cm, text centered, draw=black, fill=red!30]
        \tikzstyle{process} = [rectangle, minimum width=3cm, minimum height=1cm, text centered, draw=black, fill=orange!30, align=center]
        \tikzstyle{io} = [trapezium, trapezium left angle=70, trapezium right angle=110, minimum width=3cm, minimum height=1cm, text centered, draw=black, fill=blue!30, align=center]
        \tikzstyle{arrow} = [thick,->,>=stealth]
        
        % 节点
        \node (start) [startstop] {开始};
        \node (read_params) [io, below of=start] {读取 Q1 结果\\\(f_d\), \(f_s\), \(f_B\)};
        \node (read_audio) [io, below of=read_params] {读取音频信号 \(x(t)\)};
        \node (fft) [process, below of=read_audio] {FFT 变换\\\(X(f) = \mathcal{F}\{x(t)\}\)};
        \node (hp_filter) [process, below of=fft] {理想高通滤波\\\(X_h(f) = H_h(f) \cdot X(f)\)};
        \node (freq_shift) [process, below of=hp_filter] {频率搬移\\\(X_b(f) = \frac{1}{2}[X_h(f-f_d) + X_h(f+f_d)]\)};
        \node (lp_filter) [process, below of=freq_shift] {理想低通滤波\\\(X_l(f) = H_l(f) \cdot X_b(f)\)};
        \node (ifft) [process, below of=lp_filter] {IFFT 变换\\\(x_l(t) = \mathcal{F}^{-1}\{X_l(f)\}\)};
        \node (compare) [process, below of=ifft] {与 Q3 结果对比};
        \node (output) [io, below of=compare] {保存音频和图形};
        \node (stop) [startstop, below of=output] {结束};
        
        % 连线
        \draw [arrow] (start) -- (read_params);
        \draw [arrow] (read_params) -- (read_audio);
        \draw [arrow] (read_audio) -- (fft);
        \draw [arrow] (fft) -- (hp_filter);
        \draw [arrow] (hp_filter) -- (freq_shift);
        \draw [arrow] (freq_shift) -- (lp_filter);
        \draw [arrow] (lp_filter) -- (ifft);
        \draw [arrow] (ifft) -- (compare);
        \draw [arrow] (compare) -- (output);
        \draw [arrow] (output) -- (stop);
    \end{tikzpicture}
    \caption{Q4 频域解调流程图}
    \label{fig:q4_flowchart}
\end{figure}

\subsubsection{Q4 运行结果}

程序成功实现了频域解调,得到以下结果:

\textbf{处理参数:}
\begin{itemize}
    \item 载波频率:$f_d = 3225.1032$ Hz
    \item 采样频率:$f_s = 22050$ Hz
    \item 基带带宽:$f_B = 4000$ Hz
    \item FFT 点数:$N = 31265$
    \item 频率分辨率:$\Delta f = 0.7053$ Hz
\end{itemize}

\textbf{理想滤波器参数:}
\begin{itemize}
    \item 高通滤波器:$H_h(f) = \begin{cases} 0, & |f| < f_d \\ 1, & |f| \geq f_d \end{cases}$
    \item 低通滤波器:$H_l(f) = \begin{cases} 1, & |f| \leq f_B \\ 0, & |f| > f_B \end{cases}$
    \item 截止特性:理想砖墙型,无过渡带
\end{itemize}

\subsubsection{频谱分析}

\textbf{1. 原始信号频谱(图 \ref{fig:q4_original})}

\begin{figure}[htbp]
    \centering
    \includegraphics[width=0.9\textwidth]{codes/Q4/output/Q4_original_spectrum.png}
    \caption{Q4:原始错误解调信号频谱 $X(f)$}
    \label{fig:q4_original}
\end{figure}

原始信号频谱与 Q1 和 Q3 中的相同,主峰位于 $f = 3225.16$ Hz,这是频率偏差 $f_d$ 的位置。

\textbf{2. 理想高通滤波后频谱(图 \ref{fig:q4_xh})}

\begin{figure}[htbp]
    \centering
    \includegraphics[width=0.9\textwidth]{codes/Q4/output/Q4_xh_spectrum.png}
    \caption{Q4:理想高通滤波后信号频谱 $X_h(f)$}
    \label{fig:q4_xh}
\end{figure}

理想高通滤波器的特性:
\begin{itemize}
    \item \textbf{完美截止}:在 $f < f_d$ 处,频谱完全为零
    \item \textbf{无过渡带}:从 0 到 1 的转换是瞬时的(砖墙特性)
    \item \textbf{无衰减}:在 $f \geq f_d$ 处,信号完全保留,无任何衰减
    \item \textbf{主峰保持}:3225.16 Hz 的主峰幅度保持为 0.003444,与原始信号相同
    \item 与 Q3 的 Butterworth 滤波器相比,Q4 的理想滤波器没有过渡带,截止更加陡峭
\end{itemize}

\textbf{3. 频率搬移后频谱(图 \ref{fig:q4_xb})}

\begin{figure}[htbp]
    \centering
    \includegraphics[width=0.9\textwidth]{codes/Q4/output/Q4_xb_spectrum.png}
    \caption{Q4:频率搬移后信号频谱 $X_b(f)$}
    \label{fig:q4_xb}
\end{figure}

频率搬移通过循环移位实现:
\begin{itemize}
    \item \textbf{搬移量}:$\Delta k = \text{round}(f_d \cdot N / f_s) = 4573$ 个频率点
    \item \textbf{差频分量}:$X_h(f - f_d)$ 将信号搬移到基带(0--4000 Hz)
    \item \textbf{和频分量}:$X_h(f + f_d)$ 将信号搬移到高频(约 6450 Hz)
    \item \textbf{加权求和}:两个分量各乘以 0.5 后相加
    \item 基带峰值位于 $f = 17.63$ Hz,这是原载波信号搬移后的位置
    \item 数学等价性:频域的循环移位等价于时域的载波相乘 $x_h(t)\cos(2\pi f_d t)$
\end{itemize}

\textbf{4. 解调后信号频谱(图 \ref{fig:q4_xl})}

\begin{figure}[htbp]
    \centering
    \includegraphics[width=0.9\textwidth]{codes/Q4/output/Q4_xl_spectrum.png}
    \caption{Q4:理想低通滤波后信号频谱 $X_l(f)$ - 解调信号}
    \label{fig:q4_xl}
\end{figure}

最终解调结果:
\begin{itemize}
    \item \textbf{基带提取}:信号完全集中在 0--4000 Hz 范围内
    \item \textbf{高频抑制}:$f > f_B$ 的所有分量被理想滤波器完全滤除
    \item \textbf{主峰位置}:17.63 Hz,在基带范围内
    \item \textbf{峰值幅度}:0.000970,比 Q3 的结果略小
    \item \textbf{频谱纯净}:由于理想滤波器的完美截止特性,频谱边界非常清晰
\end{itemize}

\subsubsection{理想滤波器的数学实现}

在频域中,理想滤波器的实现非常简单直接:

\textbf{理想高通滤波器:}
\begin{equation}
    X_h[k] = H_h[k] \cdot X[k] = \begin{cases}
        0, & |f_k| < f_d \\
        X[k], & |f_k| \geq f_d
    \end{cases}
\end{equation}

其中 $f_k = k \cdot f_s / N$(对于 $k \leq N/2$)或 $f_k = (k-N) \cdot f_s / N$(对于 $k > N/2$)。

\textbf{频率搬移(循环移位):}
\begin{equation}
    X_b[k] = \frac{1}{2}\{X_h[(k + \Delta k) \bmod N] + X_h[(k - \Delta k) \bmod N]\}
\end{equation}

其中 $\Delta k = \text{round}(f_d \cdot N / f_s)$ 是频率偏移对应的频率点数。

\textbf{理想低通滤波器:}
\begin{equation}
    X_l[k] = H_l[k] \cdot X_b[k] = \begin{cases}
        X_b[k], & |f_k| \leq f_B \\
        0, & |f_k| > f_B
    \end{cases}
\end{equation}

\textbf{逆 FFT 恢复时域信号:}
\begin{equation}
    x_l[n] = \frac{1}{N} \sum_{k=0}^{N-1} X_l[k] e^{j2\pi kn/N}, \quad n = 0, 1, \ldots, N-1
\end{equation}

\subsubsection{Q3 与 Q4 对比分析}

\textbf{1. 数值比较结果}

\begin{table}[htbp]
    \centering
    \caption{Q3(时域)与 Q4(频域)解调结果对比}
    \label{tab:q3_vs_q4}
    \begin{tabular}{|l|c|c|}
        \hline
        \textbf{指标} & \textbf{Q3(时域)} & \textbf{Q4(频域)} \\
        \hline
        滤波器类型 & 8阶 Butterworth & 理想砖墙型 \\
        \hline
        截止特性 & 渐变过渡带 & 瞬时截止 \\
        \hline
        -3 dB 带宽 & 有过渡带 & 无过渡带 \\
        \hline
        相位响应 & 非线性 & 线性(无失真) \\
        \hline
        主峰频率 & 3799.95 Hz & 17.63 Hz \\
        \hline
        峰值幅度 & 0.000008 & 0.000970 \\
        \hline
        基带能量 & $6.18 \times 10^{-9}$ & $4.69 \times 10^{-5}$ \\
        \hline
        音频文件大小 & 62 KB & 62 KB \\
        \hline
    \end{tabular}
\end{table}

\textbf{2. 信号对比指标}

对两种方法得到的时域信号进行定量比较:
\begin{itemize}
    \item \textbf{均方误差(MSE)}:$4.74 \times 10^{-2}$
    \item \textbf{均方根误差(RMSE)}:$2.18 \times 10^{-1}$
    \item \textbf{最大绝对差}:0.906
    \item \textbf{相关系数}:$0.741$(中等正相关)
    \item \textbf{信噪比(SNR)}:$-21.59$ dB
\end{itemize}

\textbf{3. 相关性分析}

Q3 和 Q4 的解调结果相关系数为 $r = 0.741$(中等正相关),表明两种方法在主要信号成分上具有较好的一致性。同时,仍存在一定差异(MSE $= 4.74 \times 10^{-2}$),这是\textbf{正常且预期的},原因如下:

\begin{enumerate}
    \item \textbf{滤波器特性本质不同}
    \begin{itemize}
        \item Q3:8阶 Butterworth 滤波器,有渐变的过渡带(-48 dB/octave)
        \item Q4:理想砖墙滤波器,无过渡带,瞬时截止
    \end{itemize}
    
    \item \textbf{相位响应差异}
    \begin{itemize}
        \item Q3:IIR 滤波器引入非线性相位失真,不同频率分量的延迟不同
        \item Q4:理想滤波器本身无相位失真(但 IFFT 后可能有时域振铃)
    \end{itemize}
    
    \item \textbf{时域特性差异}
    \begin{itemize}
        \item Q3:平滑、因果的时域响应,无振铃现象
        \item Q4:由于理想滤波器的 sinc 函数冲激响应,产生 Gibbs 振铃现象
    \end{itemize}
    
    \item \textbf{实现方式差异}
    \begin{itemize}
        \item Q3:时域卷积,逐样本处理,可实时实现
        \item Q4:频域乘法,需要整个信号,批处理方式
    \end{itemize}
\end{enumerate}

\textbf{结论:}相关系数 0.741 表明两种方法在主要信号成分上具有良好的一致性,差异主要来自滤波器实现方式的本质不同。两种方法都成功地完成了 AM 信号的解调,提取出清晰可辨的音频信号。

\textbf{4. 波形对比(图 \ref{fig:q4_comparison})}

\begin{figure}[htbp]
    \centering
    \includegraphics[width=0.9\textwidth]{codes/Q4/output/Q4_vs_Q3_comparison.png}
    \caption{Q3 与 Q4 解调信号波形对比(前 2000 个采样点)}
    \label{fig:q4_comparison}
\end{figure}

从波形对比图可以观察到:
\begin{itemize}
    \item \textbf{整体趋势}:两种方法得到的信号都呈现周期性振荡
    \item \textbf{幅度差异}:Q4 的幅度变化更加剧烈,这是由于理想滤波器的砖墙特性
    \item \textbf{振铃现象}:Q4 在信号的边沿和突变处表现出明显的振铃(Gibbs 现象)
    \item \textbf{平滑度}:Q3 的波形更加平滑,因为 Butterworth 滤波器的渐变特性
\end{itemize}

\subsubsection{理想滤波器的 Gibbs 现象}

理想滤波器的砖墙特性在时域表现为 sinc 函数,这导致了著名的 Gibbs 现象:

\textbf{理论分析:}

理想低通滤波器的频率响应为矩形窗:
\begin{equation}
    H_l(f) = \text{rect}\left(\frac{f}{2f_B}\right)
\end{equation}

其时域冲激响应为 sinc 函数:
\begin{equation}
    h_l(t) = 2f_B \cdot \text{sinc}(2\pi f_B t) = \frac{\sin(2\pi f_B t)}{\pi t}
\end{equation}

该函数的特点:
\begin{itemize}
    \item \textbf{无限长}:理论上延伸到 $\pm \infty$,实际中被截断
    \item \textbf{非因果}:在 $t < 0$ 时有值,无法实时实现
    \item \textbf{振荡衰减}:以 $1/t$ 的速率衰减,但衰减较慢
    \item \textbf{振铃效应}:在信号突变处产生过冲和下冲(Gibbs 现象)
\end{itemize}

Gibbs 现象的表现:
\begin{itemize}
    \item 在矩形窗的边沿,过冲约为跳变幅度的 9\%(Gibbs 常数)
    \item 随着远离边沿,振荡幅度逐渐减小
    \item 这种振铃在 Q4 的解调信号中清晰可见
\end{itemize}

\subsubsection{方法优缺点对比}

\textbf{Q3(时域方法)优点:}
\begin{enumerate}
    \item \textbf{因果性}:可以实时处理,适合流式数据
    \item \textbf{平滑性}:Butterworth 滤波器的渐变特性避免了振铃
    \item \textbf{稳定性}:IIR 滤波器经过设计保证稳定
    \item \textbf{内存效率}:只需保存滤波器的历史状态
    \item \textbf{实用性}:可以用硬件(DSP、FPGA)实时实现
\end{enumerate}

\textbf{Q3(时域方法)缺点:}
\begin{enumerate}
    \item \textbf{非理想特性}:有过渡带,选择性不完美
    \item \textbf{相位失真}:IIR 滤波器的非线性相位可能导致波形失真
    \item \textbf{阶数限制}:过高的阶数会导致数值不稳定
    \item \textbf{设计复杂}:需要考虑截止频率、阻带衰减等多个参数
\end{enumerate}

\textbf{Q4(频域方法)优点:}
\begin{enumerate}
    \item \textbf{理想特性}:完美的频率选择性,无过渡带
    \item \textbf{无相位失真}:滤波器本身不引入相位变化
    \item \textbf{设计简单}:只需指定截止频率,直接置零即可
    \item \textbf{计算效率}:对于大信号,FFT 方法计算效率高
    \item \textbf{灵活性}:可以轻松实现任意形状的频率响应
\end{enumerate}

\textbf{Q4(频域方法)缺点:}
\begin{enumerate}
    \item \textbf{非因果性}:需要整个信号,无法实时处理
    \item \textbf{Gibbs 现象}:时域振铃,信号边沿失真
    \item \textbf{内存需求}:需要存储整个 FFT 结果
    \item \textbf{批处理}:只能离线处理,不适合流式数据
    \item \textbf{边界效应}:循环移位可能在信号边界产生不连续
\end{enumerate}

\subsubsection{应用场景选择}

\textbf{选择 Q3(时域方法)的场景:}
\begin{itemize}
    \item 实时信号处理(通信接收机、音频处理等)
    \item 流式数据处理(在线系统)
    \item 硬件实现(DSP、FPGA)
    \item 对相位失真不敏感的应用
    \item 需要因果系统的场合
\end{itemize}

\textbf{选择 Q4(频域方法)的场景:}
\begin{itemize}
    \item 离线信号分析(录音后处理、科学研究)
    \item 批处理应用(文件处理、数据分析)
    \item 需要高频率选择性的应用
    \item 对时域振铃不敏感的场合
    \item 可以接受非因果处理的应用
\end{itemize}

\subsubsection{结果验证与讨论}

\textbf{1. 频率搬移验证}

通过频谱分析验证了频率搬移的正确性:
\begin{itemize}
    \item 原始信号主峰:3225.16 Hz(载波频率)
    \item 搬移后基带峰:17.63 Hz(在基带范围内)
    \item 频率偏移:$3225.16 - 17.63 \approx 3207.53$ Hz $\approx f_d$
\end{itemize}

\textbf{2. 能量守恒分析}

\begin{table}[htbp]
    \centering
    \caption{Q4 解调过程能量分布}
    \label{tab:q4_energy}
    \begin{tabular}{|l|c|}
        \hline
        \textbf{信号阶段} & \textbf{基带能量(0--4000 Hz)} \\
        \hline
        原始信号 $X(f)$ & $4.130 \times 10^{-4}$ \\
        \hline
        理想高通 $X_h(f)$ & 约 0(主要在高频) \\
        \hline
        频率搬移 $X_b(f)$ & 中等(部分能量搬回基带) \\
        \hline
        解调信号 $X_l(f)$ & $4.690 \times 10^{-5}$ \\
        \hline
    \end{tabular}
\end{table}

能量分析表明:
\begin{itemize}
    \item 解调后基带能量约为原始基带能量的 11.4\%
    \item 能量损失主要来自高通滤波去除的低频分量
    \item Q4 保留的能量比 Q3 多得多($4.69 \times 10^{-5}$ vs $6.18 \times 10^{-9}$)
    \item 这是因为理想滤波器在通带内无衰减,而 Butterworth 滤波器有衰减
\end{itemize}

\textbf{3. 方法等价性验证}

理论上,频域的频率搬移应该等价于时域的载波相乘:
\begin{equation}
    x(t) \cos(2\pi f_d t) \overset{\mathcal{F}}{\longleftrightarrow} \frac{1}{2}[X(f-f_d) + X(f+f_d)]
\end{equation}

尽管 Q3 和 Q4 的最终波形差异较大,但它们在频域的基本操作是等价的:
\begin{itemize}
    \item 都实现了频谱从 $\pm f_d$ 到基带的搬移
    \item 都提取了 0--4000 Hz 的基带信号
    \item 差异主要来自滤波器的实现方式(非理想滤波器 vs 理想滤波器)
\end{itemize}

\textbf{4. 音频质量评估}

Q4 生成的解调音频文件特性:
\begin{itemize}
    \item 文件大小:62 KB(与 Q3 相同)
    \item 采样参数:22050 Hz, 16-bit, 单声道
    \item 时长:1.42 秒
    \item 主观听感:可能有轻微的边缘失真(Gibbs 振铃)
    \item 整体信号质量良好,可正常播放,且可以辨别出原音频内容为"Enjoy your spring break!"
\end{itemize}

\subsubsection{理论与实践的对比}

\begin{table}[htbp]
    \centering
    \caption{Q4 理论与实际实现对比}
    \label{tab:q4_theory_practice}
    \begin{tabular}{|l|l|l|}
        \hline
        \textbf{项目} & \textbf{理论} & \textbf{实际实现} \\
        \hline
        高通滤波器 & 理想砖墙 & 数字理想砖墙 \\
        \hline
        低通滤波器 & 理想砖墙 & 数字理想砖墙 \\
        \hline
        频率搬移 & 连续频率搬移 & 离散循环移位 \\
        \hline
        FFT/IFFT & 连续傅里叶变换 & 离散傅里叶变换 \\
        \hline
        截止特性 & 瞬时(无限陡) & 单个频率点截止 \\
        \hline
        时域冲激响应 & 无限长 sinc & 有限长近似 \\
        \hline
        Gibbs 现象 & 理论上存在 & 实际中可观察到 \\
        \hline
        因果性 & 非因果 & 非因果(批处理) \\
        \hline
    \end{tabular}
\end{table}

主要差异:
\begin{itemize}
    \item \textbf{离散化}:实际实现使用 DFT 而非连续傅里叶变换
    \item \textbf{有限精度}:浮点运算有舍入误差
    \item \textbf{频率分辨率}:受限于 FFT 点数($\Delta f = 0.7053$ Hz)
    \item \textbf{时域截断}:信号长度有限,冲激响应被自然截断
\end{itemize}

尽管存在这些差异,实际实现很好地近似了理论模型,解调效果符合预期。

\section{总结与体会}

通过本次工程设计,我深入研究了调幅(AM)信号解调的理论与实践,完成了从频谱分析、滤波器设计到时域和频域解调的完整实现。这个过程不仅巩固了信号与系统课程的理论知识,更让我体会到了理论与实践结合的重要性。

\subsection{技术收获}

\subsubsection{1. 信号处理理论的深入理解}

在完成四个子问题的过程中,我对信号处理的核心概念有了更深刻的认识:

\textbf{频域分析的重要性:}通过 Q1 的频谱分析,我认识到频域是理解信号特性的关键视角。时域信号可能看起来复杂无规律,但在频域中,其结构往往一目了然。通过 FFT 变换和频谱可视化,我成功估计出载波频率偏差 $f_d = 3225.1032$ Hz,这为后续的解调工作奠定了基础。这让我体会到,\textbf{选择正确的分析域往往是解决问题的关键}。

\textbf{滤波器设计的工程权衡:}Q2 中设计 8 阶 Butterworth 滤波器时,我深刻体会到了理论与实践的差距。理想滤波器虽然性能完美,但无法实现;而实际滤波器需要在截止陡度、阻带衰减、通带平坦度、相位线性度等多个指标间权衡。选择 Butterworth 滤波器是因为其最大平坦幅度特性,8 阶设计则是在性能和复杂度间的折中。这让我理解了\textbf{工程设计没有完美方案,只有最合适的方案}。

\textbf{调制解调的对偶性:}Q3 和 Q4 分别用时域和频域方法实现解调,让我深刻认识到傅里叶变换的对偶性。时域的载波相乘对应频域的频谱搬移,时域的卷积对应频域的乘法。这种对偶关系不仅是数学上的优美,更为实际问题提供了多种解决途径。\textbf{同一个问题可以有完全不同的实现方式,各有优劣}。

\subsubsection{2. 编程能力的提升}

本次设计中,我选择使用 Rust 语言实现所有算法,这是一次充满挑战但收获颇丰的经历:

\textbf{Rust 语言特性的应用:}Rust 的所有权系统、类型安全和零成本抽象为信号处理提供了强大的保障。在处理大规模音频数据时,不用担心内存泄漏或野指针问题;编译器的严格检查帮助我在开发阶段就发现了许多潜在 bug。虽然学习曲线陡峭,但掌握后的开发效率和程序可靠性都很高。

\textbf{模块化设计:}我将每个子问题划分为多个功能模块(如音频读写、滤波器、频谱分析、解调器等),每个模块职责单一、接口清晰。这种设计不仅使代码易于理解和维护,也便于在不同问题间复用代码。例如,\texttt{spectrum\_analyzer.rs} 模块在 Q1、Q3、Q4 中都被复用。

\textbf{第三方库的使用:}学会了如何选择和使用优秀的第三方库。\texttt{hound} 用于 WAV 文件处理,\texttt{rustfft} 实现高效 FFT,\texttt{plotters} 生成专业图表,\texttt{num-complex} 处理复数运算。这些库都经过良好测试,使用它们大大提高了开发效率。这让我认识到\textbf{站在巨人的肩膀上,专注于问题本身而非重复造轮子}。

\textbf{调试与优化:}在实现过程中遇到了许多实际问题,如频率搬移方向错误、滤波器系数计算精度、FFT 结果归一化等。通过仔细分析中间结果、对比理论值、编写验证程序等方法逐一解决。特别是在高通滤波器设计中,采用频谱反转法(Spectral Inversion)代替传统的模拟域变换,实现了高精度的截止频率控制(误差 $< 0.01\%$),使得 Q3 和 Q4 的解调结果相关性达到 0.741,验证了两种方法的正确性。

这些实践经历让我深刻认识到:\textbf{验证是确保正确性的关键,系统性思考是解决问题的基础}。要通过多种方法验证结果,确保算法实现符合理论预期。

\subsubsection{3. 理论与实践的结合}

\textbf{理论指导实践:}课本上的公式和定理为实现提供了明确的方向。例如,Butterworth 滤波器的传递函数、傅里叶变换对、卷积定理等,都直接对应代码实现。没有扎实的理论基础,很难正确实现这些算法。

\textbf{实践验证理论:}另一方面,实践也加深了对理论的理解。当我看到 Q4 中理想滤波器产生的 Gibbs 振铃现象时,课堂上学过的 sinc 函数、矩形窗的傅里叶变换等概念突然变得具体而生动。\textbf{理论告诉我们"是什么"和"为什么",实践让我们看到"长什么样"}。

\textbf{工程约束的考虑:}理论中的"理想"在实践中往往无法实现。Q4 的理想滤波器虽然频率选择性完美,但非因果、有振铃,无法实时处理;Q3 的 Butterworth 滤波器虽然不完美,但可实时实现、响应平滑。这让我认识到\textbf{工程实践需要在理想与现实间找到平衡}。

\subsection{方法论收获}

\subsubsection{1. 问题分解与模块化}

面对复杂的工程问题,我从命题老师给出的4个小问中学会了将其分解为若干子问题并逐步解决一个复杂问题的流程与方法:
\begin{itemize}
    \item Q1:分析问题,确定参数
    \item Q2:设计工具,准备滤波器
    \item Q3:实现方案一,时域解调
    \item Q4:实现方案二,频域解调
\end{itemize}

每个子问题又可进一步分解为更小的模块。这种\textbf{自顶向下、逐步细化}的方法使复杂问题变得可控。

\subsubsection{2. 对比与验证}

在整个设计中,我始终注重对比与验证:
\begin{itemize}
    \item 频域结果与时域结果对比(FFT 的正确性)
    \item 理论值与实测值对比(算法的准确性)
    \item 不同方法的结果对比(Q3 vs Q4)
    \item 中间过程的可视化(每个处理阶段的频谱)
\end{itemize}

这种\textbf{多角度验证}的思维方式提高了结果的可信度,也帮助我发现和修正了多处错误。

\subsubsection{3. 文档与可视化}

好的工程不仅要有正确的结果,还要有清晰的表达。我通过以下方式增强了设计的可读性:
\begin{itemize}
    \item \textbf{流程图}:直观展示算法流程
    \item \textbf{频谱图}:可视化信号在各处理阶段的变化
    \item \textbf{对比表}:量化不同方法的性能差异
    \item \textbf{详细注释}:代码中的关键步骤都有说明
    \item \textbf{结果文件}:将关键数据保存为文本,便于查阅
\end{itemize}

这让我认识到\textbf{表达能力与技术能力同样重要}。

\subsection{对 AM 解调的深入认识}

\subsubsection{1. 解调的本质}

AM 解调的本质是\textbf{频谱搬移 + 滤波}。调制时,基带信号的频谱被搬移到载波附近;解调则是将其搬回基带,并滤除不需要的分量。这个过程可以在时域实现(载波相乘 + 低通滤波),也可以在频域实现(频谱移位 + 理想滤波)。

\subsubsection{2. 频率偏差的影响}

本次设计中,原始信号存在频率偏差 $f_d = 3225.1032$ Hz,这导致解调不当时信号分布在错误的频率范围。通过 Q1 的精确估计,Q3 和 Q4 都能正确补偿这个偏差。这说明\textbf{准确的参数估计是后续处理的基础}。

\subsubsection{3. 时域与频域的选择}

两种解调方法各有千秋:
\begin{itemize}
    \item \textbf{时域方法(Q3)}:可实时处理,响应平滑,易于硬件实现,适合通信系统
    \item \textbf{频域方法(Q4)}:频率选择性好,设计简单,适合离线分析,但有振铃
\end{itemize}

实际应用中应根据具体需求选择合适的方法。

\subsection{遇到的挑战与解决}

\subsubsection{1. Rust 语言的学习曲线}

\textbf{挑战}:Rust 的所有权系统、生命周期、借用规则等概念对初学者不友好,编译器错误信息虽然详细但需要时间理解。

\textbf{解决}:通过阅读官方文档《The Rust Programming Language》、查阅示例代码、在编译器提示下反复修改,逐步掌握了 Rust 的核心概念。虽然初期困难,但后期开发效率很高。

\subsubsection{2. FFT 的归一化问题}

\textbf{挑战}:不同 FFT 库的归一化方式不同,\texttt{rustfft} 的 FFT 不包含归一化因子,IFFT 也不包含,需要手动除以 $N$。

\textbf{解决}:仔细阅读库文档,理解其实现细节,在 IFFT 后手动除以样本数进行归一化。同时在 Q4 中添加了 2 倍增益补偿,使结果与理论一致。

\subsubsection{3. 频率搬移的方向问题}

\textbf{挑战}:Q4 中频率搬移最初实现方向错误,导致基带信号位置不对。

\textbf{解决}:通过绘制中间结果的频谱图,发现问题所在。将 \texttt{output[shifted\_idx] = spectrum[i]} 改为 \texttt{result[i] = spectrum[shifted\_idx]},即从源数组的偏移位置读取,而非写入到偏移位置。

\subsubsection{4. 高通滤波器设计错误的发现与修复}

\textbf{挑战}:Q3 和 Q4 的解调结果相关系数仅为 -0.028,远低于预期,让我怀疑实现有误。

\textbf{初步分析}:最初认为低相关性是由于 Butterworth 滤波器与理想滤波器的特性差异导致,例如:
\begin{itemize}
    \item Butterworth 滤波器有渐变的过渡带,理想滤波器是砖墙截止
    \item IIR 滤波器的非线性相位失真
    \item 理想滤波器的 Gibbs 振铃现象
\end{itemize}

但这些差异通常只会导致波形细节不同,不应该造成负相关。

\textbf{深入调查}:
\begin{enumerate}
    \item 编写测试程序 \texttt{test\_cutoff.rs},验证滤波器实际截止频率
    \item 发现高通滤波器的实际 -3 dB 截止频率为 10946 Hz,而设计值是 3225 Hz
    \item 误差高达 239\%,说明滤波器设计存在严重问题
    \item 定位到问题根源:原设计采用模拟域 $s \to \omega_c^2/s$ 变换设计高通滤波器,在双线性变换时产生严重的频率扭曲
    \item 这导致 Q3 处理的频率范围完全错误,与 Q4 的理想滤波器处理的频率范围不同
\end{enumerate}

\textbf{解决方案}:
\begin{enumerate}
    \item 改用频谱反转法(Spectral Inversion)设计高通滤波器
    \item 先设计镜像截止频率 $f_c' = f_s/2 - f_c = 7799.9$ Hz 的低通滤波器
    \item 对低通滤波器系数的奇数索引项取反:$b_i' = (-1)^i b_i$,$a_i' = (-1)^i a_i$
    \item 得到 $H_{HP}(z) = H_{LP}(-z)$,实现频谱关于 $f_s/4$ 的镜像
    \item 重新验证:修复后的高通滤波器截止频率误差 $< 0.01\%$
\end{enumerate}

\textbf{效果验证}:
\begin{itemize}
    \item Q3 和 Q4 的相关系数从 -0.028 提升至 0.741(中等正相关)
    \item 两种方法现在都正确地处理相同的频率范围
    \item 剩余的差异(MSE $= 4.74 \times 10^{-2}$)主要来自滤波器实现方式的本质不同:
    \begin{itemize}
        \item 渐变过渡带 vs 砖墙截止
        \item 非线性相位 vs 线性相位
        \item 平滑响应 vs Gibbs 振铃
    \end{itemize}
    \item 两种方法都成功解调出清晰可辨的音频信号
\end{itemize}

\textbf{经验教训}:
\begin{itemize}
    \item \textbf{定量验证至关重要}:不能仅通过听音频或观察波形就认为算法正确,必须通过定量测试验证关键参数
    \item \textbf{异常结果需深入排查}:当遇到异常结果时,要系统性地排查各个环节,而非简单归因于"方法不同"
    \item \textbf{理论与实践的结合}:模拟域的理论公式在数字域实现时可能产生意外的扭曲,需要选择更适合数字域的方法
    \item \textbf{测试驱动开发}:编写专门的测试程序验证算法的关键特性,是发现问题的有效手段
\end{itemize}

\subsection{不足与改进方向}

尽管完成了设计要求,但仍有一些不足之处:

\subsubsection{1. 滤波器阶数的优化}

Q2 中直接选择了 8 阶 Butterworth 滤波器,但没有系统地分析不同阶数的性能差异。未来可以绘制阶数与性能指标(过渡带宽度、阻带衰减、相位非线性度等)的关系曲线,找到最优阶数。

\subsubsection{2. 实时处理性能}

当前实现都是批处理方式,没有考虑实时性。Q3 的时域方法理论上可以实时处理,但需要改造为流式架构。未来可以实现一个实时音频处理系统,测试实际处理延迟。

\subsubsection{3. 窗函数的应用}

频谱分析时使用了矩形窗,会产生频谱泄漏。未来可以尝试 Hamming、Hanning、Blackman 等窗函数,减少频谱泄漏,提高频率估计精度。

\subsubsection{4. 更多解调方法的对比}

除了同步解调(本次实现的方法),AM 解调还有包络检波等方法。未来可以实现多种方法并对比其性能、复杂度、适用场景等。

\subsubsection{5. 抗噪声性能分析}

当前信号较为理想,未来可以在信号中加入不同强度的噪声,测试各种方法的抗噪声性能,这对实际通信系统更有意义。

\subsection{对未来学习和工作的启示}

这次工程设计给我带来了许多启示,将影响我未来的学习和工作:

\textbf{1. 扎实的理论基础是创新的前提}

没有信号与系统、数字信号处理等课程的理论知识,很难完成本次设计。未来无论从事什么工作,都要重视基础知识的学习,因为\textbf{理论是解决复杂问题的钥匙}。

\textbf{2. 动手实践是检验真理的标准}

课堂上学的知识只有通过实践才能真正掌握。本次设计让我对许多理论概念有了直观认识,远比单纯做习题印象深刻。\textbf{纸上得来终觉浅,绝知此事要躬行}。

\textbf{3. 工具的选择很重要}

选择合适的编程语言、开发工具、第三方库能大大提高效率。本次选择 Rust 虽然学习成本高,但带来了类型安全、高性能、零成本抽象等优势。\textbf{磨刀不误砍柴工,好工具是效率的保证}。

\textbf{4. 系统思维与全局观}

工程问题往往涉及多个环节,需要从整体角度思考。本次设计中,Q1 的频率估计精度影响 Q3、Q4 的解调效果;Q2 的滤波器性能影响 Q3 的信号质量。\textbf{局部最优不等于全局最优,要有系统思维}。

\textbf{5. 沟通与表达的重要性}

技术能力固然重要,但如果不能清晰表达,价值就会大打折扣。本次报告的撰写过程让我认识到\textbf{技术文档的质量与技术本身同样重要},好的文档能让别人快速理解你的工作。

\textbf{6. 追求卓越但接受不完美}

工程设计永远没有完美方案,总是在多个约束条件下寻找最优解。本次设计中,Q3 和 Q4 各有优劣,没有绝对的好坏。\textbf{接受不完美,在约束下追求卓越,这就是工程的魅力}。

\subsection{实践体会}

本次工程设计是一次宝贵的学习经历,让我在理论与实践的结合中成长。虽然过程中遇到了许多困难,但克服这些困难的过程本身就是最大的收获。我相信,这次经历将成为我学术道路上的重要一步,激励我在信号处理、通信系统等领域继续探索。

\textbf{纸上得来终觉浅,绝知此事要躬行。}这是本次设计最深刻的体会。


% 参考文献
\newpage
\begin{thebibliography}{99}
    \addcontentsline{toc}{section}{参考文献}
    \bibitem {ref1} Alan V. Oppenheim, Alan S. Willsky. 信号与系统(第 2 版)[M]. 北京:电子工业出版社,2013.
    \bibitem{ref2} 郑君里,应启珩,杨为理. 信号与系统(第三版)[M]. 北京:高等教育出版社,2011
    \bibitem{ref3} RustFFT Documentation. https://docs.rs/rustfft/, 2023
    \bibitem{ref4} Plotters Documentation. https://docs.rs/plotters/, 2023
    \bibitem {ref5} 李钟慎。基于 MATLAB 设计巴特沃斯低通滤波器 [J]. 信息技术,2003,27(3):49--50.
    \bibitem {ref6} 祝广场,李志,梅映新。基于 Matlab 的巴特沃斯滤波器设计 [J]. 船电技术,2012,32(B08):28--30.
    \bibitem {ref7} 杨丽娟,张白桦,叶旭桢,等。快速傅里叶变换 FFT 及其应用 [J]. Opto-Electronic Engineering,2004,31(z1):1--3.
    \bibitem {ref8} 丁志中。双线性变换法原理的解释 [J]. 电气电子教学学报,2004,26(2):53--54.
    \bibitem {ref9} 张登奇,彭鑫,陈海兰。双线性变换法在 IIR 滤波器设计中的应用 [J]. 湖南理工学院学报:自然科学版,2016,29(3):21--25.
\end{thebibliography}

% 附录
\newpage
\appendix

\section{快速傅里叶变换(FFT)简介}
\addcontentsline{toc}{section}{附录A 快速傅里叶变换简介}

\subsection{FFT的基本原理}

快速傅里叶变换(Fast Fourier Transform, FFT)是一种高效计算离散傅里叶变换(DFT)的算法。对于长度为$N$的序列$x[n]$,其DFT定义为:

\begin{equation}
X[k] = \sum_{n=0}^{N-1} x[n] e^{-j\frac{2\pi}{N}kn}, \quad k=0,1,\ldots,N-1
\end{equation}

直接计算DFT需要$O(N^2)$次复数乘法,而FFT算法将复杂度降低到$O(N\log N)$,极大地提高了计算效率。

\subsection{Cooley-Tukey算法}

最常用的FFT算法是Cooley-Tukey算法,采用分治策略(Divide and Conquer)。当$N$为2的幂次时,可以递归地将DFT分解为两个长度为$N/2$的DFT:

\begin{equation}
X[k] = \sum_{n=0}^{N/2-1} x[2n] e^{-j\frac{2\pi}{N}k(2n)} + \sum_{n=0}^{N/2-1} x[2n+1] e^{-j\frac{2\pi}{N}k(2n+1)}
\end{equation}

定义旋转因子$W_N = e^{-j\frac{2\pi}{N}}$,可以写成:

\begin{equation}
X[k] = E[k] + W_N^k O[k]
\end{equation}

其中$E[k]$和$O[k]$分别是偶数项和奇数项的$N/2$点DFT。利用对称性:

\begin{align}
X[k] &= E[k] + W_N^k O[k], \quad k=0,1,\ldots,\frac{N}{2}-1 \\
X[k+\frac{N}{2}] &= E[k] - W_N^k O[k], \quad k=0,1,\ldots,\frac{N}{2}-1
\end{align}

这样每次递归将问题规模减半,递归深度为$\log_2 N$,总计算量为$O(N\log N)$。

\subsection{FFT的性质}

\subsubsection{1. 线性性质}
\begin{equation}
\mathcal{F}\{ax[n] + by[n]\} = aX[k] + bY[k]
\end{equation}

\subsubsection{2. 时移性质}
\begin{equation}
\mathcal{F}\{x[n-n_0]\} = e^{-j\frac{2\pi}{N}kn_0}X[k]
\end{equation}

\subsubsection{3. 频移性质}
\begin{equation}
\mathcal{F}\{e^{j\frac{2\pi}{N}k_0n}x[n]\} = X[k-k_0]
\end{equation}

\subsubsection{4. 卷积定理}
时域卷积对应频域相乘:
\begin{equation}
\mathcal{F}\{x[n] * h[n]\} = X[k] \cdot H[k]
\end{equation}

频域卷积对应时域相乘:
\begin{equation}
\mathcal{F}\{x[n] \cdot h[n]\} = \frac{1}{N}X[k] \ast H[k]
\end{equation}

\subsubsection{5. Parseval定理}
\begin{equation}
\sum_{n=0}^{N-1} |x[n]|^2 = \frac{1}{N}\sum_{k=0}^{N-1} |X[k]|^2
\end{equation}

\subsection{FFT在本项目中的应用}

\subsubsection{频谱分析(Q1)}
通过FFT将时域信号$x[n]$变换到频域,得到频谱$X[k]$:
\begin{itemize}
    \item 计算幅度谱:$|X[k]| = \sqrt{\text{Re}^2(X[k]) + \text{Im}^2(X[k])}$
    \item 寻找峰值频率:$f_d = \arg\max_k |X[k]| \cdot \frac{f_s}{N}$
    \item 频率分辨率:$\Delta f = \frac{f_s}{N} = 0.7053$ Hz
\end{itemize}

\subsubsection{频域滤波(Q4)}
在频域实现滤波,避免时域卷积的高复杂度:
\begin{enumerate}
    \item 对信号进行FFT:$X[k] = \text{FFT}\{x[n]\}$
    \item 频域相乘:$Y[k] = X[k] \cdot H[k]$(理想滤波器)
    \item 逆FFT返回时域:$y[n] = \text{IFFT}\{Y[k]\}$
\end{enumerate}

频域滤波的复杂度为$O(N\log N)$,而时域卷积为$O(N^2)$或$O(N \cdot M)$($M$为滤波器长度)。

\subsubsection{频率搬移(Q4)}
通过循环移位实现频率搬移:
\begin{equation}
X_{\text{shifted}}[k] = \frac{1}{2}[X[(k+\Delta k) \bmod N] + X[(k-\Delta k) \bmod N]]
\end{equation}

其中$\Delta k = \lfloor f_d \cdot N / f_s \rfloor$为频率偏移对应的频点数。

\subsection{FFT实现细节}

\subsubsection{基-2 FFT}
要求$N = 2^m$,本项目中$N = 31265$不是2的幂,FFT库自动补零到$N = 32768 = 2^{15}$。

\subsubsection{位反转(Bit-Reversal)}
FFT算法中需要对输入进行位反转排序:
\begin{itemize}
    \item 索引0(二进制:000)→ 0(二进制:000)
    \item 索引1(二进制:001)→ 4(二进制:100)
    \item 索引2(二进制:010)→ 2(二进制:010)
    \item 索引3(二进制:011)→ 6(二进制:110)
\end{itemize}

\subsubsection{原位计算(In-Place)}
FFT可以在原数组上进行,无需额外存储空间,节省内存。

\subsubsection{归一化}
不同FFT库的归一化方式不同:
\begin{itemize}
    \item \textbf{前向归一化}:FFT除以$N$,IFFT不除
    \item \textbf{后向归一化}:FFT不除,IFFT除以$N$
    \item \textbf{对称归一化}:FFT和IFFT都除以$\sqrt{N}$
\end{itemize}

本项目使用rustfft库,采用后向归一化方式。

\subsection{FFT的优化技巧}

\subsubsection{1. SIMD向量化}
现代FFT库使用SSE/AVX指令集并行处理多个数据。

\subsubsection{2. 缓存优化}
按块处理数据,提高缓存命中率。

\subsubsection{3. 混合基FFT}
支持$N = 2^a \cdot 3^b \cdot 5^c$等分解,不局限于2的幂。

\subsubsection{4. 实数FFT优化}
对于实数输入,可以利用对称性减少一半计算量。

\newpage
\section{Butterworth滤波器简介}
\addcontentsline{toc}{section}{附录B Butterworth滤波器简介}

\subsection{Butterworth滤波器的基本概念}

Butterworth滤波器是一种最大平坦幅度滤波器(Maximally Flat Magnitude Filter),由英国工程师Stephen Butterworth于1930年提出。其主要特点是通带内幅频响应最平坦,无纹波,过渡带单调下降。

\subsection{模拟Butterworth滤波器}

\subsubsection{幅频响应}

$n$阶模拟Butterworth低通滤波器的幅频响应为:

\begin{equation}
|H_a(j\omega)|^2 = \frac{1}{1 + (\omega/\omega_c)^{2n}}
\end{equation}

其中:
\begin{itemize}
    \item $\omega_c$:截止频率(-3dB频率)
    \item $n$:滤波器阶数
    \item $\omega$:角频率(rad/s)
\end{itemize}

在截止频率处:$|H_a(j\omega_c)| = \frac{1}{\sqrt{2}} \approx 0.707$(-3dB)

\subsubsection{关键特性}

\textbf{1. 通带最大平坦性}

在$\omega = 0$处,前$2n-1$阶导数为零,保证通带内最大平坦:
\begin{equation}
\left.\frac{d^k|H_a(j\omega)|^2}{d\omega^k}\right|_{\omega=0} = 0, \quad k=1,2,\ldots,2n-1
\end{equation}

\textbf{2. 单调下降}

阻带内幅频响应单调递减,无纹波起伏。

\textbf{3. 衰减特性}

高频衰减率为$20n$ dB/decade或$6n$ dB/octave,阶数越高衰减越快。

\subsubsection{极点分布}

Butterworth滤波器的极点均匀分布在单位圆上,满足:

\begin{equation}
s_k = \omega_c e^{j\theta_k}, \quad \theta_k = \frac{\pi(2k+n-1)}{2n}, \quad k=1,2,\ldots,n
\end{equation}

对于低通滤波器,选择左半平面的极点($\text{Re}(s_k) < 0$)保证系统稳定。

$n=8$时,极点角度为:
\begin{equation}
\theta_k = \frac{\pi(2k+7)}{16}, \quad k=1,2,\ldots,8
\end{equation}

即:$\theta_1 = 9\pi/16$,$\theta_2 = 11\pi/16$,...,$\theta_8 = 23\pi/16$

\subsubsection{传递函数}

模拟Butterworth低通滤波器的传递函数为:

\begin{equation}
H_a(s) = \frac{\omega_c^n}{\prod_{k=1}^{n}(s - s_k)}
\end{equation}

分母是Butterworth多项式,具有特殊的递归结构。

\subsection{数字Butterworth滤波器设计}

\subsubsection{双线性变换法}

将模拟滤波器转换为数字滤波器,使用双线性变换:

\begin{equation}
s = \frac{2}{T}\frac{1-z^{-1}}{1+z^{-1}} = \frac{2f_s(1-z^{-1})}{1+z^{-1}}
\end{equation}

其中$T = 1/f_s$为采样周期,$f_s$为采样频率。

\textbf{频率映射关系:}
\begin{equation}
\omega = \frac{2}{T}\tan\left(\frac{\Omega}{2}\right) = 2f_s\tan\left(\frac{\pi f}{f_s}\right)
\end{equation}

其中$\Omega = 2\pi f / f_s$为数字角频率。

\subsubsection{预畸变}

由于双线性变换的非线性映射,需要对截止频率进行预畸变:

\begin{equation}
\omega_c = 2f_s\tan\left(\frac{\pi f_c}{f_s}\right)
\end{equation}

\textbf{本项目参数:}
\begin{itemize}
    \item 高通滤波器:$f_c = 3225.1032$ Hz,$\omega_c = 2 \times 22050 \times \tan(\pi \times 3225.1032 / 22050) \approx 21475.89$ rad/s
    \item 低通滤波器:$f_c = 4000$ Hz,$\omega_c = 2 \times 22050 \times \tan(\pi \times 4000 / 22050) \approx 26682.17$ rad/s
\end{itemize}

\subsubsection{高通滤波器设计}

高通滤波器可以通过两种方法设计:

\textbf{方法1:模拟域变换}

从低通原型变换为高通滤波器,使用变换:

\begin{equation}
s_{\text{LP}} \rightarrow \frac{\omega_c^2}{s_{\text{HP}}}
\end{equation}

或在频域:$\omega_{\text{LP}} = \omega_c^2 / \omega_{\text{HP}}$

\textbf{方法2:频谱反转法(本项目采用)}

在数字域直接从低通滤波器转换:

\begin{equation}
H_{\text{HP}}(z) = H_{\text{LP}}(-z)
\end{equation}

具体步骤:
\begin{enumerate}
    \item 设计镜像截止频率为 $f_c' = f_s/2 - f_c$ 的低通滤波器
    \item 对滤波器系数的奇数索引项取反:
    \begin{equation}
    b_i' = (-1)^i b_i, \quad a_i' = (-1)^i a_i
    \end{equation}
    \item 这样得到的高通滤波器截止频率为 $f_c$
\end{enumerate}

频谱反转法的优势:
\begin{itemize}
    \item 避免模拟域的 $s \to \omega_c^2/s$ 变换,减少频率扭曲
    \item 实现简单,只需修改系数符号
    \item 数值稳定性好,截止频率精度高(误差 $< 0.01\%$)
\end{itemize}

\subsection{数字滤波器传递函数}

经过双线性变换后,得到数字滤波器的传递函数:

\begin{equation}
H(z) = \frac{b_0 + b_1z^{-1} + b_2z^{-2} + \cdots + b_nz^{-n}}{a_0 + a_1z^{-1} + a_2z^{-2} + \cdots + a_nz^{-n}}
\end{equation}

归一化使$a_0 = 1$:

\begin{equation}
H(z) = \frac{b_0 + b_1z^{-1} + \cdots + b_nz^{-n}}{1 + a_1z^{-1} + \cdots + a_nz^{-n}}
\end{equation}

对于8阶滤波器,$b$和$a$分别有9个系数。

\subsection{频率响应}

数字滤波器的频率响应为:

\begin{equation}
H(e^{j\omega}) = H(z)|_{z=e^{j\omega}} = \frac{\sum_{k=0}^{n}b_k e^{-j\omega k}}{\sum_{k=0}^{n}a_k e^{-j\omega k}}
\end{equation}

其中$\omega = 2\pi f / f_s$为归一化角频率。

\textbf{幅频响应:}
\begin{equation}
|H(e^{j\omega})| = \left|\frac{\sum_{k=0}^{n}b_k e^{-j\omega k}}{\sum_{k=0}^{n}a_k e^{-j\omega k}}\right|
\end{equation}

\textbf{相频响应:}
\begin{equation}
\angle H(e^{j\omega}) = \arg\left(\frac{\sum_{k=0}^{n}b_k e^{-j\omega k}}{\sum_{k=0}^{n}a_k e^{-j\omega k}}\right)
\end{equation}

\subsection{本项目中的Butterworth滤波器}

\subsubsection{设计参数}

\textbf{高通滤波器:}
\begin{itemize}
    \item 阶数:$n = 8$
    \item 截止频率:$f_c = 3225.1032$ Hz(估计的载波频率$f_d$)
    \item 归一化截止频率:$\omega_c = 2\pi f_c / f_s = 0.9186$ rad
    \item 功能:抑制低频直流分量和低频噪声
\end{itemize}

\textbf{低通滤波器:}
\begin{itemize}
    \item 阶数:$n = 8$
    \item 截止频率:$f_c = 4000$ Hz(基带带宽$f_B$)
    \item 归一化截止频率:$\omega_c = 2\pi f_c / f_s = 1.1395$ rad
    \item 功能:提取基带信号,抑制高频分量
\end{itemize}

\subsubsection{性能特点}

\textbf{1. 通带平坦}
\begin{itemize}
    \item 通带内增益接近1(0dB),无纹波
    \item 在$f=0$(低通)或$f=f_s/2$(高通)处增益最大
\end{itemize}

\textbf{2. 单调过渡}
\begin{itemize}
    \item 从通带到阻带单调递减
    \item 8阶滤波器衰减率:48 dB/octave
\end{itemize}

\textbf{3. 相位特性}
\begin{itemize}
    \item 非线性相位,存在相位失真
    \item 群延迟在截止频率附近较大
    \item 不适合要求严格相位线性的应用
\end{itemize}

\textbf{4. 稳定性}
\begin{itemize}
    \item 所有极点位于单位圆内,系统稳定
    \item IIR结构,反馈系数$a_k$决定稳定性
\end{itemize}

\subsection{IIR滤波器实现}

\subsubsection{Direct Form II结构}

Direct Form II是最节省存储空间的IIR滤波器实现结构,状态方程为:

\begin{align}
w[n] &= x[n] - \sum_{k=1}^{n}a_k w[n-k] \\
y[n] &= \sum_{k=0}^{n}b_k w[n-k]
\end{align}

只需要$n$个状态变量$w[n-1], w[n-2], \ldots, w[n-n]$。

\subsubsection{数值稳定性}

\textbf{量化误差:}
\begin{itemize}
    \item 系数量化影响极点位置,可能导致不稳定
    \item 使用双精度浮点数(64位)减少量化误差
\end{itemize}

\textbf{溢出问题:}
\begin{itemize}
    \item 递归计算可能导致数值溢出
    \item 需要适当缩放输入或中间结果
\end{itemize}

\subsection{与其他滤波器的比较}

\begin{table}[htbp]
    \centering
    \caption{常见滤波器特性比较}
    \begin{tabular}{|l|c|c|c|}
        \hline
        \textbf{滤波器类型} & \textbf{通带特性} & \textbf{过渡带} & \textbf{相位特性} \\
        \hline
        Butterworth & 最大平坦,无纹波 & 单调,较宽 & 非线性 \\
        \hline
        Chebyshev I & 等纹波 & 陡峭 & 非线性 \\
        \hline
        Chebyshev II & 平坦 & 陡峭 & 非线性 \\
        \hline
        Elliptic & 等纹波 & 最陡峭 & 非线性 \\
        \hline
        Bessel & 平坦 & 较宽 & 近似线性 \\
        \hline
    \end{tabular}
\end{table}

\textbf{Butterworth的优势:}
\begin{itemize}
    \item 通带最平坦,适合不希望信号失真的应用
    \item 设计简单,参数调节方便
    \item 性能平衡,通带、阻带和过渡带特性适中
\end{itemize}

\textbf{Butterworth的劣势:}
\begin{itemize}
    \item 过渡带较宽,选择性不如Chebyshev和Elliptic
    \item 相位非线性,存在相位失真
    \item 阶数较高时计算复杂度增加
\end{itemize}

\subsection{应用场景}

\begin{itemize}
    \item \textbf{音频处理}:通带平坦,不引入明显失真
    \item \textbf{信号解调}:如本项目的AM解调,平滑滤除不需要的频率成分
    \item \textbf{抗混叠滤波}:ADC前置滤波器
    \item \textbf{平滑处理}:去除测量数据中的高频噪声
\end{itemize}

\newpage
\section{程序主要代码}
\addcontentsline{toc}{section}{附录C 程序主要代码}

本次设计使用 Rust 语言实现。完整代码已上传至 GitHub 仓库,供复现核验:

\begin{center}
\texttt{https://github.com/yzy-pro/HUST\_SignalandSystemBigHomework}
\end{center}

\subsection{项目结构}

代码按问题划分为四个独立的 Rust 工程,结构如下:

\begin{verbatim}
codes/
├── Q1/                      # Frequency spectrum analysis
│   ├── src/
│   │   ├── main.rs          # Main program
│   │   ├── audio_reader.rs  # WAV file reading
│   │   ├── spectrum_analyzer.rs  # FFT and spectrum
│   │   └── plotter.rs       # Plotting functions
│   ├── Cargo.toml           # Dependencies
│   └── output/              # Results (4 PNG + 1 TXT)
│
├── Q2/                      # Butterworth filter design
│   ├── src/
│   │   ├── main.rs
│   │   ├── filter_design.rs # Filter coefficient calculation
│   │   └── plotter.rs
│   ├── Cargo.toml
│   └── output/              # Results (7 PNG + 2 TXT)
│
├── Q3/                      # Time-domain demodulation
│   ├── src/
│   │   ├── main.rs
│   │   ├── audio_reader.rs
│   │   ├── iir_filter.rs    # Direct Form II IIR
│   │   ├── demodulator.rs   # Synchronous demodulation
│   │   ├── spectrum_analyzer.rs
│   │   └── audio_writer.rs  # WAV file writing
│   ├── Cargo.toml
│   └── output/              # Results (4 PNG + 1 WAV + 2 TXT)
│
└── Q4/                      # Frequency-domain demodulation
    ├── src/
    │   ├── main.rs
    │   ├── audio_reader.rs
    │   ├── ideal_filter.rs  # Ideal brick-wall filters
    │   ├── frequency_shifter.rs  # Circular frequency shift
    │   ├── spectrum_analyzer.rs
    │   ├── audio_writer.rs
    │   └── comparator.rs    # Q3 vs Q4 comparison
    ├── Cargo.toml
    └── output/              # Results (4 PNG + 1 WAV + 3 TXT)
\end{verbatim}

\subsection{主要依赖库}

本项目使用的核心依赖库:

\begin{table}[htbp]
    \centering
    \caption{Rust 依赖库列表}
    \begin{tabular}{|l|l|p{7cm}|}
        \hline
        \textbf{库名称} & \textbf{版本} & \textbf{功能} \\
        \hline
        hound & 3.5 & WAV 音频文件读写 \\
        \hline
        rustfft & 6.1 & 高效FFT/IFFT实现 \\
        \hline
        plotters & 0.3.1 & 专业数据可视化和图表生成 \\
        \hline
        num-complex & 0.4 & 复数运算支持 \\
        \hline
    \end{tabular}
\end{table}

\subsection{编译与运行}

每个子项目都是独立的 Rust 工程,可单独编译运行,具体代码复现方法如下:

\begin{verbatim}
cd codes/Q1 # 进入子项目目录

cargo build --release # 编译

cargo run --release # 运行(输出文件在 output/ 目录下)
\end{verbatim}

\textbf{环境依赖:}
\begin{itemize}
    \item Rust 1.75.0 或更高版本
    \item Cargo 包管理器
    \item Linux-ubuntu24 操作系统
\end{itemize}

\subsection{核心算法实现要点}

\subsubsection{Q1 - FFT频谱分析}
\begin{itemize}
    \item 使用 \texttt{rustfft} 库的 \texttt{FftPlanner} 进行FFT变换
    \item 实现峰值检测算法,在正频率部分寻找最大幅度
    \item 频率分辨率:$\Delta f = f_s / N = 0.7053$ Hz
    \item 输出四种频谱图:全频段、低频段、dB刻度、时域波形
\end{itemize}

\subsubsection{Q2 - Butterworth滤波器设计}
\begin{itemize}
    \item 模拟滤波器极点计算:$s_k = \omega_c e^{j\theta_k}$,$\theta_k = \frac{\pi(2k+n+1)}{2n}$
    \item 双线性变换:$s = \frac{2}{T}\frac{1-z^{-1}}{1+z^{-1}}$
    \item 数字滤波器系数:9个分子系数、9个分母系数
    \item 频率响应计算:$H(e^{j\omega}) = \frac{\sum b_k z^{-k}}{\sum a_k z^{-k}}$
\end{itemize}

\subsubsection{Q3 - 时域解调}
\begin{itemize}
    \item Direct Form II IIR滤波器实现,状态空间方程
    \item 载波相乘:$x_b[n] = 2 \cdot x_h[n] \cdot \cos(2\pi f_d n / f_s)$
    \item 增益补偿因子:2.0(补偿相乘后的能量衰减)
    \item 逐样本处理,支持实时流式计算
\end{itemize}

\subsubsection{Q4 - 频域解调}
\begin{itemize}
    \item 理想高通滤波器:$H_h(f) = \begin{cases} 0, & |f| < f_d \\ 1, & |f| \geq f_d \end{cases}$
    \item 循环移位实现频率搬移:$X_b[k] = 0.5[X_h[(k+\Delta k) \bmod N] + X_h[(k-\Delta k) \bmod N]]$
    \item 理想低通滤波器:$H_l(f) = \begin{cases} 1, & |f| \leq f_B \\ 0, & |f| > f_B \end{cases}$
    \item IFFT归一化:除以样本数$N$,并乘以2倍增益补偿
\end{itemize}

\subsection{输出文件说明}

\subsubsection{Q1 输出 (5个文件)}
\begin{itemize}
    \item \texttt{Q1\_spectrum\_full.png} - 全频段频谱图(0-11025 Hz)
    \item \texttt{Q1\_spectrum\_lowfreq.png} - 低频段频谱图(0-5000 Hz)
    \item \texttt{Q1\_spectrum\_db.png} - dB刻度频谱图
    \item \texttt{Q1\_waveform.png} - 时域波形图(前2000个采样点)
    \item \texttt{Q1\_results.txt} - 分析结果文本($f_d$估计值等)
\end{itemize}

\subsubsection{Q2 输出 (9个文件)}
\begin{itemize}
    \item 7个PNG图表:HP/LP的幅频响应、幅频响应(dB)、相频响应、组合响应
    \item \texttt{Q2\_hp\_coefficients.txt} - 高通滤波器系数
    \item \texttt{Q2\_lp\_coefficients.txt} - 低通滤波器系数
\end{itemize}

\subsubsection{Q3 输出 (7个文件)}
\begin{itemize}
    \item 4个PNG频谱图:原始信号、高通后、相乘后、解调后
    \item \texttt{Q3\_demodulated.wav} - 解调后音频文件(22050 Hz, 16-bit)
    \item \texttt{Q3\_results.txt} - 各阶段频谱峰值频率
    \item \texttt{Q3\_summary.txt} - 解调性能总结
\end{itemize}

\subsubsection{Q4 输出 (8个文件)}
\begin{itemize}
    \item 4个PNG频谱图:原始信号、理想HP后、频移后、理想LP后
    \item \texttt{Q4\_demodulated.wav} - 解调后音频文件
    \item \texttt{Q4\_results.txt} - 各阶段频谱峰值频率
    \item \texttt{Q4\_comparison.txt} - Q3 vs Q4 对比指标
    \item \texttt{Q4\_vs\_Q3\_comparison.png} - 波形对比图
\end{itemize}

\subsection{完整源代码}

以下是各模块的完整源代码。所有代码文件均位于 \texttt{codes/} 目录下,按问题编号分类组织。

\subsubsection{Q1 - 频谱分析与频率估计}

\paragraph{主程序 (main.rs)}
\lstinputlisting[
    language=Rust, 
    caption={Q1主程序 - 频谱分析流程控制}, 
    basicstyle=\ttfamily\scriptsize,
    breaklines=true
]{codes/Q1/main.rs}

\paragraph{音频读取模块 (audio\_reader.rs)}
\lstinputlisting[
    language=Rust, 
    caption={音频文件读取 - WAV格式解析}, 
    basicstyle=\ttfamily\scriptsize,
    breaklines=true
]{codes/Q1/audio_reader.rs}

\paragraph{FFT处理模块 (fft\_processor.rs)}
\lstinputlisting[
    language=Rust, 
    caption={FFT变换处理 - 使用rustfft库}, 
    basicstyle=\ttfamily\scriptsize,
    breaklines=true
]{codes/Q1/fft_processor.rs}

\paragraph{频率估计模块 (frequency\_estimator.rs)}
\lstinputlisting[
    language=Rust, 
    caption={载波频率估计 - 峰值检测算法}, 
    basicstyle=\ttfamily\scriptsize,
    breaklines=true
]{codes/Q1/frequency_estimator.rs}

\paragraph{频谱可视化模块 (spectrum\_visualizer.rs)}
\lstinputlisting[
    language=Rust, 
    caption={频谱图绘制 - 使用plotters库}, 
    basicstyle=\ttfamily\scriptsize,
    breaklines=true
]{codes/Q1/spectrum_visualizer.rs}

\subsubsection{Q2 - Butterworth滤波器设计}

\paragraph{主程序 (main.rs)}
\lstinputlisting[
    language=Rust, 
    caption={Q2主程序 - 滤波器设计流程}, 
    basicstyle=\ttfamily\scriptsize,
    breaklines=true
]{codes/Q2/main.rs}

\paragraph{滤波器设计模块 (butterworth\_filter.rs)}
\lstinputlisting[
    language=Rust, 
    caption={Butterworth滤波器设计 - 双线性变换法}, 
    basicstyle=\ttfamily\scriptsize,
    breaklines=true
]{codes/Q2/butterworth_filter.rs}

\paragraph{响应可视化模块 (response\_visualizer.rs)}
\lstinputlisting[
    language=Rust, 
    caption={频率响应可视化 - 幅频/相频特性}, 
    basicstyle=\ttfamily\scriptsize,
    breaklines=true
]{codes/Q2/response_visualizer.rs}

\subsubsection{Q3 - 时域解调}

\paragraph{主程序 (main.rs)}
\lstinputlisting[
    language=Rust, 
    caption={Q3主程序 - 时域同步解调流程}, 
    basicstyle=\ttfamily\scriptsize,
    breaklines=true
]{codes/Q3/src/main.rs}

\paragraph{音频读取模块 (audio\_reader.rs)}
\lstinputlisting[
    language=Rust, 
    caption={音频文件读取模块}, 
    basicstyle=\ttfamily\scriptsize,
    breaklines=true
]{codes/Q3/src/audio_reader.rs}

\paragraph{IIR滤波器模块 (iir\_filter.rs)}
\lstinputlisting[
    language=Rust, 
    caption={Direct Form II IIR滤波器实现}, 
    basicstyle=\ttfamily\scriptsize,
    breaklines=true
]{codes/Q3/src/iir_filter.rs}

\paragraph{解调器模块 (demodulator.rs)}
\lstinputlisting[
    language=Rust, 
    caption={时域同步解调 - 载波相乘法}, 
    basicstyle=\ttfamily\scriptsize,
    breaklines=true
]{codes/Q3/src/demodulator.rs}

\paragraph{频谱分析模块 (spectrum\_analyzer.rs)}
\lstinputlisting[
    language=Rust, 
    caption={频谱分析 - FFT与峰值检测}, 
    basicstyle=\ttfamily\scriptsize,
    breaklines=true
]{codes/Q3/src/spectrum_analyzer.rs}

\paragraph{音频写入模块 (audio\_writer.rs)}
\lstinputlisting[
    language=Rust, 
    caption={WAV文件写入 - 16bit PCM格式}, 
    basicstyle=\ttfamily\scriptsize,
    breaklines=true
]{codes/Q3/src/audio_writer.rs}

\subsubsection{Q4 - 频域解调}

\paragraph{主程序 (main.rs)}
\lstinputlisting[
    language=Rust, 
    caption={Q4主程序 - 频域解调完整流程}, 
    basicstyle=\ttfamily\scriptsize,
    breaklines=true
]{codes/Q4/src/main.rs}

\paragraph{音频读取模块 (audio\_reader.rs)}
\lstinputlisting[
    language=Rust, 
    caption={音频文件读取模块}, 
    basicstyle=\ttfamily\scriptsize,
    breaklines=true
]{codes/Q4/src/audio_reader.rs}

\paragraph{理想滤波器模块 (ideal\_filter.rs)}
\lstinputlisting[
    language=Rust, 
    caption={理想高通/低通滤波器 - 频域矩形窗}, 
    basicstyle=\ttfamily\scriptsize,
    breaklines=true
]{codes/Q4/src/ideal_filter.rs}

\paragraph{频率搬移模块 (frequency\_shifter.rs)}
\lstinputlisting[
    language=Rust, 
    caption={循环移位频率搬移 - 频谱平移}, 
    basicstyle=\ttfamily\scriptsize,
    breaklines=true
]{codes/Q4/src/frequency_shifter.rs}

\paragraph{频谱分析模块 (spectrum\_analyzer.rs)}
\lstinputlisting[
    language=Rust, 
    caption={频谱分析模块}, 
    basicstyle=\ttfamily\scriptsize,
    breaklines=true
]{codes/Q4/src/spectrum_analyzer.rs}

\paragraph{音频写入模块 (audio\_writer.rs)}
\lstinputlisting[
    language=Rust, 
    caption={WAV文件写入模块}, 
    basicstyle=\ttfamily\scriptsize,
    breaklines=true
]{codes/Q4/src/audio_writer.rs}

\paragraph{对比分析模块 (comparator.rs)}
\lstinputlisting[
    language=Rust, 
    caption={Q3与Q4结果对比 - 性能分析}, 
    basicstyle=\ttfamily\scriptsize,
    breaklines=true
]{codes/Q4/src/comparator.rs}

\end{document}